% \LaTeX-Main\
% !TeX encoding=UTF-8
% !TeX spellcheck=en_US
%% The LaTeX package csvsimple - version 2.7.0 (2024/09/27)
%% csvsimple.tex: Manual
%%
%% -------------------------------------------------------------------------------------------
%% Copyright (c) 2008-2024 by Prof. Dr. Dr. Thomas F. Sturm <thomas dot sturm at unibw dot de>
%% -------------------------------------------------------------------------------------------
%%
%% This work may be distributed and/or modified under the
%% conditions of the LaTeX Project Public License, either version 1.3
%% of this license or (at your option) any later version.
%% The latest version of this license is in
%%   http://www.latex-project.org/lppl.txt
%% and version 1.3 or later is part of all distributions of LaTeX
%% version 2005/12/01 or later.
%%
%% This work has the LPPL maintenance status `author-maintained'.
%%
%% This work consists of all files listed in README.md
%%
% \RequirePackage[check-declarations,enable-debug]{expl3}
\documentclass[a4paper,11pt]{ltxdoc}
\usepackage{csvsimple-doc}

\usepackage{\csvpkgprefix csvsimple-l3}

\tcbmakedocSubKey[doc key prefix=]{docCsvKey}{csvsim}
\tcbmakedocSubKeys[doc key prefix=]{docCsvKeys}{csvsim}

\hypersetup{
  pdftitle={Manual for the csvsimple-l3 package},
  pdfauthor={Thomas F. Sturm},
  pdfsubject={csv file processing with LaTeX3},
  pdfkeywords={csv file, comma separated values, key value syntax}
}

\usepackage{incgraph}

%%%%%%%%%%%%%%%%%%%%%%%%%%%%%%%%%%%%%%%%%%%%%%%%%
\begin{document}


\begin{center}
\begin{tcolorbox}[enhanced,hbox,tikznode,left=8mm,right=8mm,boxrule=0.4pt,
  colback=white,colframe=black!50!yellow,
  drop lifted shadow=black!50!yellow,arc is angular,
  before=\par\vspace*{5mm},after=\par\bigskip]
{\bfseries\LARGE The \texttt{csvsimple-l3} package}\\[3mm]
{\large Manual for version \version\ (\datum)}
\end{tcolorbox}
{\large Thomas F.~Sturm%
  \footnote{Prof.~Dr.~Dr.~Thomas F.~Sturm, Institut f\"{u}r Mathematik und Informatik,
    University of the Bundeswehr Munich, D-85577 Neubiberg, Germany;
    email: \href{mailto:thomas.sturm@unibw.de}{thomas.sturm@unibw.de}}\par\medskip
\normalsize\url{https://www.ctan.org/pkg/csvsimple}\par
\url{https://github.com/T-F-S/csvsimple}
}
\end{center}
\bigskip
\begin{absquote}
  \begin{center}\bfseries Abstract\end{center}
  |csvsimple(-l3)| provides a simple \LaTeX\ interface for the processing of files with
  comma separated values (CSV). |csvsimple-l3| relies heavily on the key value
  syntax from |l3keys| which results in an easy way of usage.
  Filtering and table generation is especially supported. Since the package
  is considered as a lightweight tool, there is no support for data sorting
  or data base storage.
\end{absquote}

\vspace{1cm}

\includegraphics[width=\linewidth]{csvsimple-title.png}
% Source code for the title picture - omitted for PDF viewer compatibility
\begin{tcolorbox}[void]
\begin{NoHyper}
\begin{inctext}[]
\begin{tikzpicture}
\fill[top color=blue!50!gray!50,bottom color=red!50!gray!50]  (-8,-5) rectangle (8,5);
\node at (0,2.5) {\tcbinputlisting{listing file=csvsimple-example.csv,listing only,width=11cm,blankest,colupper=blue!50!black}};
\node[red!50!black] at (0,-2.5) {\csvautotabular{csvsimple-example.csv}};
\begin{scope}[transparency group=knockout]
\fill [top color=blue!50!gray!10,bottom color=red!50!gray!10] (-7.7,-4.7) rectangle (7.7,4.7);
\node at (0,2.5) {\tcbinputlisting{listing file=csvsimple-example.csv,listing only,width=11cm,blankest,colupper=blue!20}};
\node[red!20] at (0,-2.5) {\csvautotabular{csvsimple-example.csv}};
\node at (0,2.5) [opacity=0,font=\fontencoding{T1}\fontfamily{lmr}\fontsize{7cm}{7cm}\bfseries] {csv};
\node at (0,-2.5) [opacity=0,font=\fontencoding{T1}\fontfamily{lmr}\fontsize{4.8cm}{4.8cm}\bfseries] {simple};
\end{scope}
\end{tikzpicture}
\end{inctext}
\end{NoHyper}
\end{tcolorbox}


\clearpage
\tableofcontents

\clearpage
\section{Introduction}%
The |csvsimple-l3| package is applied to the processing of
CSV\footnote{CSV file: file with comma separated values.} files.
This processing is controlled by key value assignments according to the
syntax of |l3keys|. Sample applications of the package
are tabular lists, serial letters, and charts.

An alternative to |csvsimple-l3| is the \ctanpkg{datatool} package
which provides considerably more functions and allows sorting of data by \LaTeX.
|csvsimple-l3| has a different approach for the user interface and
is deliberately restricted to some basic functions with fast
processing speed.

Mind the following restrictions:
\begin{itemize}
\item Sorting is not supported directly but can be done
  with external tools, see \Fullref{sec:Sorting}.
\item Values are expected to be comma separated, but the package
  provides support for other separators, see \Fullref{sec:separators}.
\item Values are expected to be either not quoted or quoted with
  curly braces |{}| of \TeX\ groups. Other quotes like double-quotes
  are not supported directly, but can be achieved
  with external tools, see \Fullref{sec:importeddata}.
  For approximate patching see \Fullref{sec:hooks}.
\item Every data line is expected to contain the same amount of values.
  Unfeasible data lines are silently ignored by default, but this can
  be configured, see \Fullref{sec:consistency}.
\end{itemize}


\subsection{Loading the Package}
|csvsimple-l3| is loaded with \emph{one} of the following
alternatives inside the preamble:
\begin{dispListing}
\usepackage[l3]{csvsimple}
   % or alternatively (not simultaneously!)
\usepackage{csvsimple-l3}
\end{dispListing}

Not automatically loaded, but used for many examples are the packages
\ctanpkg{longtable}, \ctanpkg{booktabs}, \ctanpkg{ifthen}, and \ctanpkg{etoolbox}.


\clearpage
\subsection{First Steps}
Every line of a processable CSV file has to contain an identical amount of
comma\footnote{See \refKey{csvsim/separator} for other separators than comma.} separated values. The curly braces |{}| of \TeX\ groups can be used
to mask a block which may contain commas not to be processed as separators.

The first line of such a CSV file is usually but not necessarily a header line
which contains the identifiers for each column.

%-- file embedded for simplicity --
\begin{tcbverbatimwrite}{grade.csv}
name,givenname,matriculation,gender,grade
Maier,Hans,12345,m,1.0
Huber,Anna,23456,f,2.3
Weißbäck,Werner,34567,m,5.0
Bauer,Maria,19202,f,3.3
\end{tcbverbatimwrite}
%-- end embedded file --

\csvlisting{grade}

\smallskip
The most simple way to display a CSV file in tabular form is the processing
with the \refCom{csvautotabular} command.

\begin{dispExample}
\csvautotabular{grade.csv}
\end{dispExample}


Typically, one would use \refCom{csvreader} instead of |\csvautotabular| to
gain full control over the interpretation of the included data.

In the following example, the entries of the header line are automatically
assigned to \TeX\ macros which may be used deliberately.


\begin{dispExample}
\begin{tabular}{|l|c|}\hline%
\bfseries Person & \bfseries Matr.~No.
\csvreader[
    head to column names
  ]{grade.csv}{}{%
    \\\givenname\ \name & \matriculation
  }%
\\\hline
\end{tabular}
\end{dispExample}


\clearpage
|\csvreader| is controlled by a plenty of options. For example, for table
applications line breaks are easily inserted by
\refKey{csvsim/late after line}. This defines a macro execution just before
the following line.
Additionally, the assignment of columns to \TeX\ macros is shown in a non automated
way.

\begin{dispExample}
\begin{tabular}{|r|l|c|}\hline%
& Person & Matr.~No.\\\hline\hline
\csvreader[
    late after line = \\\hline
  ]{grade.csv}%
  {name=\name, givenname=\firstname, matriculation=\matnumber}{%
    \thecsvrow & \firstname~\name & \matnumber
  }%
\end{tabular}
\end{dispExample}

\smallskip
An even more comfortable and preferable way to create a table is setting
appropriate option keys. Note, that this gives you the possibility to create a
meta key (called style here) which contains the whole table creation
using \refCom{csvstyle} or |\keys_define:nn| from |l3keys|.

\begin{dispExample}
\csvreader[
    tabular         = |r|l|c|,
    table head      = \hline & Person & Matr.~No.\\\hline\hline,
    late after line = \\\hline
  ]{grade.csv}
  {name=\name, givenname=\firstname, matriculation=\matnumber}{%
    \thecsvrow & \firstname~\name & \matnumber
  }%
\end{dispExample}


\clearpage
The next example shows such a style definition with the convenience macro
\refCom{csvstyle}. Here, we see again the automated assignment of header
entries to column names by \refKey{csvsim/head to column names}.
For this, the header entries have to be without spaces and special characters.
But you can always assign entries to canonical macro names manually like in the examples
above. Here, we also add a \refKey{csvsim/head to column names prefix} to avoid
macro name clashes.

\begin{dispExample}
\csvstyle{myTableStyle}{
    tabular         = |r|l|c|,
    table head      = \hline & Person & Matr.~No.\\\hline\hline,
    late after line = \\\hline,
    head to column names,
    head to column names prefix = MY,
  }

\csvreader[myTableStyle]
  {grade.csv}{}{%
    \thecsvrow & \MYgivenname~\MYname & \MYmatriculation
  }
\end{dispExample}


\smallskip
Another way to address columns is to use their roman numbers.
The direct addressing is done by |\csvcoli|, |\csvcolii|, |\csvcoliii|, \ldots:

\begin{dispExample}
\csvreader[
    tabular         = |r|l|c|,
    table head      = \hline & Person & Matr.~No.\\\hline\hline,
    late after line = \\\hline
  ]{grade.csv}{}{%
    \thecsvrow & \csvcolii~\csvcoli & \csvcoliii
  }
\end{dispExample}

\smallskip
And yet another method to assign macros to columns is to use arabic numbers
for the assignment:

\begin{dispExample}
\csvreader[
    tabular         = |r|l|c|,
    table head      = \hline & Person & Matr.~No.\\\hline\hline,
    late after line = \\\hline]%
  {grade.csv}
  {1=\name, 2=\firstname, 3=\matnumber}{%
    \thecsvrow & \firstname~\name & \matnumber
  }
\end{dispExample}

\smallskip
For recurring applications, the |l3keys| syntax allows to create own meta options
(styles) for a consistent and centralized design. The following example is easily
modified to obtain more or less option settings.

\begin{dispExample}
\csvstyle{myStudentList}{%
    tabular         = |r|l|c|,
    table head      = \hline & Person & #1\\\hline\hline,
    late after line = \\\hline,
    column names    = {name=\name, givenname=\firstname}
  }

\csvreader[ myStudentList={Matr.~No.} ]
  {grade.csv}
  {matriculation=\matnumber}{%
    \thecsvrow & \firstname~\name & \matnumber
  }%
\hfill%
\csvreader[ myStudentList={Grade} ]
  {grade.csv}
  {grade=\grade}{%
    \thecsvrow & \firstname~\name & \grade
  }
\end{dispExample}


\clearpage
Alternatively, column names can be set by \refCom{csvnames}
and style definitions by \refCom{csvstyle}.
With this, the last example is rewritten as follows:

\begin{dispExample}
\csvnames{myNames}{1=\name,2=\firstname,3=\matnumber,5=\grade}
\csvstyle{myStudentList}{
    tabular         = |r|l|c|,
    table head      = \hline & Person & #1\\\hline\hline,
    late after line = \\\hline,
    myNames
  }

\csvreader[ myStudentList={Matr.~No.} ]
  {grade.csv}{}{%
    \thecsvrow & \firstname~\name & \matnumber
  }%
\hfill%
\csvreader[ myStudentList={Grade} ]
  {grade.csv}{}{%
    \thecsvrow & \firstname~\name & \grade
  }
\end{dispExample}

\smallskip
The data lines of a CSV file can also be filtered. In the following example,
a certificate is printed only for students with grade unequal to 5.0.

\begin{dispExample}
\csvreader[
    filter not strcmp={\grade}{5.0}
  ]{grade.csv}
  {1=\name,2=\firstname,3=\matnumber,4=\gender,5=\grade}{%
    \begin{center}\Large\bfseries Certificate in Mathematics\end{center}
    \large\IfCsvsimStrEqualTF{\gender}{f}{Ms.}{Mr.}
    \firstname~\name, matriculation number \matnumber, has passed the test
    in mathematics with grade \grade.\par\ldots\par
  }%
\end{dispExample}


\clearpage
\section{Macros for the Processing of CSV Files}\label{sec:makros}%

\begin{docCommand}{csvreader}{\oarg{options}\marg{file name}\marg{assignments}\marg{command list}}
  \refCom{csvreader} reads the file denoted by \meta{file name} line by line.
  Every line of the file has to contain an identical amount of
  comma separated values. The curly braces |{}| of \TeX\ groups can be used
  to mask a block which may contain commas not to be processed as separators.\smallskip

  The first line of such a CSV file is by default but not necessarily
  processed as a header line which contains the identifiers for each column.
  The entries of this line can be used to give \meta{assignments} to \TeX\ macros
  to address the columns. The number of entries of this first line
  determines the accepted number of entries for all following lines.
  Every line which contains a higher or lower number of entries is ignored
  during standard processing.\smallskip

  The \meta{assignments} are given as comma separated list of key value pairs
  \mbox{\meta{name}|=|\meta{macro}}. Here, \meta{name} is an entry from the
  header line \emph{or} the arabic number of the addressed column.
  \meta{macro} is some \TeX\ macro which gets the content of the addressed column.\smallskip

  The \meta{command list} is executed for every accepted data line. Inside the
  \meta{command list} is applicable:
  \begin{itemize}
  \item \docAuxCommand{thecsvrow} or the counter |csvrow| which contains the number of the
    current data line (starting with 1).
  \item \docAuxCommand{csvcoli}, \docAuxCommand{csvcolii}, \docAuxCommand{csvcoliii}, \ldots,
    which contain the contents of the column entries of the current data line.
    Alternatively can be used:
  \item \meta{macro} from the \meta{assignments} to have a logical
    addressing of a column entry.
  \end{itemize}
  Note, that the \meta{command list} is allowed to contain |\par| and
  that \textbf{all macro definitions are made global} to be used for table applications.\smallskip

  The processing of the given CSV file can be controlled by various
  \meta{options} given as key value list. The feasible option keys
  are described in section \ref{sec:schluessel} from page \pageref{sec:schluessel}.

\begin{dispExample}
\csvreader[
    tabular    = |r|l|l|,
    table head = \hline,
    table foot = \hline
  ]{grade.csv}%
  {name=\name, givenname=\firstname, grade=\grade}{%
    \grade & \firstname~\name & \csvcoliii
  }
\end{dispExample}

Mainly, the |\csvreader| command consists of a \refCom{csvloop} macro with
following parameters:\par
|\csvloop{|\meta{options}|, file=|\meta{file name}|, column names=|\meta{assignments}|,|\\
  \hspace*{2cm} |command=|\meta{command list}|}|\par
  Therefore, the application of the keys \refKey{csvsim/file} and \refKey{csvsim/command}
is useless for |\csvreader|.
\end{docCommand}


\clearpage
\begin{docCommand}{csvloop}{\marg{options}}
  Usually, \refCom{csvreader} may be preferred instead of |\csvloop|.
  \refCom{csvreader} is based on |\csvloop| which takes a mandatory list of
  \meta{options} in key value syntax.
  This list of \meta{options} controls the total processing. Especially,
  it has to contain the CSV file name.
\begin{dispExample}
\csvloop{
    file                 = {grade.csv},
    head to column names,
    command              = \name,
    before reading       = {List of students:\ },
    late after line      = {{,}\ },
    late after last line = .
  }
\end{dispExample}
\end{docCommand}


\begin{docCommand}[doc updated = 2021-06-25]{csvset}{\marg{options}}
  Sets \meta{options} for every following
  \refCom{csvreader} and \refCom{csvloop}.
  Note that most options are set to default values at the begin of these
  commands and therefore cannot be defined reasonable by \refCom{csvset}.
  But it may be used for options like \refKey{csvsim/csvsorter command}
  to give global settings. Also see \refKey{csvsim/every csv}.
\end{docCommand}


\begin{docCommand}{csvstyle}{\marg{key}\marg{options}}
  Defines a new |l3keys| meta key to call other keys. It is used to
  make abbreviations for convenient key set applications.
  The new \meta{key} can take one parameter. The name \refCom{csvstyle}
  originates from an old version of |csvsimple| which used |pgfkeys|
  instead of |l3keys|.

\begin{dispExample}
\csvstyle{grade list}{
    column names = {name=\name, givenname=\firstname, grade=\grade}
  }
\csvstyle{passed}{
    filter not strcmp = {\grade}{5.0}
  }
The following students passed the test in mathematics:\\
\csvreader[grade list,passed]{grade.csv}{}{
   \firstname\ \name\ (\grade);
  }
\end{dispExample}
\end{docCommand}

\enlargethispage*{1cm}

\begin{docCommand}{csvnames}{\marg{key}\marg{assignments}}
  Abbreviation for |\csvstyle{|\meta{key}|}{column names=|\marg{assignments}|}|
  to define additional \meta{assignments} of macros to columns.
\begin{dispExample}
\csvnames{grade list}{
    name=\name, givenname=\firstname, grade=\grade
  }
\csvstyle{passed}{
    filter not strcmp = {\grade}{5.0}
  }
The following students passed the test in mathematics:\\
\csvreader[grade list,passed]{grade.csv}{}{
   \firstname\ \name\ (\grade);
  }
\end{dispExample}
\end{docCommand}


%\begin{docCommand}{csvheadset}{\marg{assignments}}
%  For some special cases, this command can be used to change the
%  \meta{assignments} of macros to columns during execution of
%  \refCom{csvreader} and \refCom{csvloop}.
%\begin{dispExample}
%\csvreader{grade.csv}{}%
%  { \csvheadset{name=\n} \fbox{\n}
%    \csvheadset{givenname=\n} \ldots\ \fbox{\n}  }%
%\end{dispExample}
%\end{docCommand}

\clearpage


\begin{docCommand}[doc updated=2021-06-28]{ifcsvoddrow}{\marg{then macros}\marg{else macros}}
  Inside the command list of \refCom{csvreader}, the \meta{then macros}
  are executed for odd-numbered   data lines, and the \meta{else macros}
  are executed for even-numbered lines.
  \refCom{ifcsvoddrow} is expandable.
\begin{dispExample}
\csvreader[
    head to column names,
    tabular    = |l|l|l|l|,
    table head = \hline\bfseries \# & \bfseries Name & \bfseries Grade\\\hline,
    table foot = \hline
  ]{grade.csv}{}{%
    \ifcsvoddrow{\slshape\thecsvrow & \slshape\name, \givenname & \slshape\grade}%
    {\bfseries\thecsvrow & \bfseries\name, \givenname & \bfseries\grade}
  }
\end{dispExample}

The |\ifcsvoddrow| macro may be used for striped tables:

\begin{dispExample}
% This example needs the xcolor package
\csvreader[
    head to column names,
    tabular    = rlcc,
    table head = \hline\rowcolor{red!50!black}\color{white}\# & \color{white}Person
      & \color{white}Matr.~No. & \color{white}Grade,
    late after head = \\\hline\rowcolor{yellow!50},
    late after line = \ifcsvoddrow{\\\rowcolor{yellow!50}}{\\\rowcolor{red!25}}
  ]{grade.csv}{}{%
    \thecsvrow & \givenname~\name & \matriculation & \grade
  }
\end{dispExample}

Alternatively, |\rowcolors| from the |xcolor| package can be used for this
purpose:

\begin{dispExample}
% This example needs the xcolor package
\csvreader[
    head to column names,
    tabular      = rlcc,
    before table = \rowcolors{2}{red!25}{yellow!50},
    table head   = \hline\rowcolor{red!50!black}\color{white}\# & \color{white}Person
      & \color{white}Matr.~No. & \color{white}Grade\\\hline
  ]{grade.csv}{}{%
    \thecsvrow & \givenname~\name & \matriculation & \grade
  }
\end{dispExample}

  The deprecated, but still available alias for this command is
  \docAuxCommand{csvifoddrow}.
\end{docCommand}

\clearpage

\begin{docCommand}[doc updated=2021-06-28]{ifcsvfirstrow}{\marg{then macros}\marg{else macros}}
  Inside the command list of \refCom{csvreader}, the \meta{then macros}
  are executed for the first data line, and the \meta{else macros}
  are executed for all following lines.
  \refCom{ifcsvfirstrow} is expandable.
\begin{dispExample}
\csvreader[
    tabbing,
    head to column names,
    table head = {\hspace*{3cm}\=\kill}
  ]{grade.csv}{}{%
    \givenname~\name \> (\ifcsvfirstrow{first entry!!}{following entry})
  }
\end{dispExample}
  The deprecated, but still available alias for this command is
  \docAuxCommand{csviffirstrow}.
\end{docCommand}

\medskip


\begin{docCommand}{csvfilteraccept}{}
  All following consistent data lines will be accepted and processed.
  This command overwrites all previous filter settings and may be used
  inside \refKey{csvsim/full filter} to implement
  an own filtering rule together with |\csvfilterreject|.
\begin{dispExample}
\csvreader[
    autotabular,
    full filter = \IfCsvsimStrEqualTF{\csvcoliv}{m}{\csvfilteraccept}{\csvfilterreject}
  ]{grade.csv}{}{%
    \csvlinetotablerow
  }
\end{dispExample}
\end{docCommand}


\begin{docCommand}{csvfilterreject}{}
  All following data lines will be ignored.
  This command overwrites all previous filter settings.
\end{docCommand}


\begin{docCommand}{csvline}{}
  This macro contains the current and unprocessed data line.
\begin{dispExample}
\csvreader[
    no head,
    tabbing,
    table head = {\textit{line XX:}\=\kill}
  ]{grade.csv}{}{%
    \textit{line \thecsvrow:} \> \csvline
  }
\end{dispExample}
\end{docCommand}


\clearpage
\begin{docCommand}[doc updated=2022-01-11]{csvlinetotablerow}{}
  Typesets the current processed data line with |&| between the entries.
  This macro is \emph{expandable}.
\end{docCommand}


\begin{docCommands}{
    { doc name = thecsvrow       , doc updated = 2021-06-25 },
    { doc name = g_csvsim_row_int, doc new     = 2021-06-25 }
  }
  Typesets the current data line number. This is the
  current number of accepted data lines without the header line.
  Despite of the name, there is no associated \LaTeX\ counter |csvrow|,
  but \refCom{thecsvrow} accesses the \LaTeX3 integer
  \refCom{g_csvsim_row_int}.
\end{docCommands}


\begin{docCommands}[doc new=2021-06-25]{
    { doc name = thecsvcolumncount },
    { doc name = g_csvsim_columncount_int }
  }
  Typesets the number of columns of the current CSV file. This number
  is either computed from the first valid line (header or data) or
  given by \refKey{csvsim/column count}.
  Despite of the name, there is no associated \LaTeX\ counter |csvcolumncount|,
  but \refCom{thecsvcolumncount} accesses the \LaTeX3 integer
  \refCom{g_csvsim_columncount_int}.
\begin{dispExample}
\csvreader{grade.csv}{}{}%
The last file consists of \thecsvcolumncount{} columns and
\thecsvrow{} accepted data lines. The total number of lines
is \thecsvinputline{}.
\end{dispExample}
\end{docCommands}


\begin{docCommands}{
    { doc name = thecsvinputline       , doc updated = 2021-06-25 },
    { doc name = g_csvsim_inputline_int, doc new     = 2021-06-25 }
  }
  Typesets the current file line number. This is the
  current number of all data lines including the header line and all
  lines filtered out.
  Despite of the name, there is no associated \LaTeX\ counter |csvinputline|,
  but \refCom{thecsvinputline} accesses the \LaTeX3 integer
  \refCom{g_csvsim_inputline_int}.
\begin{dispExample}
\csvreader[
    no head,
    filter test = \ifnumequal{\thecsvinputline}{3}
  ]{grade.csv}{}{%
    The line with number \thecsvinputline\ contains: \csvline
  }
\end{dispExample}
\end{docCommands}



\clearpage
\section{Macros for Automatic Survey Tables}\label{sec:autotable}%

The following |\csvauto...| commands are intended for quick data overview
with \emph{limited} formatting potential.
The most customizable |\csvauto...| commands are
\refCom{csvautotabularray} and friends.

For full control see Subsection~\ref{subsec:tabsupport} on page \pageref{subsec:tabsupport}
for the general table options in combination with \refCom{csvreader} and
\refCom{csvloop}.

\begin{docCommands}[
    doc parameter = \oarg{options}\marg{file name}
  ]
  {
    { doc name = csvautotabular  },
    { doc name = csvautotabular*, doc new = 2021-06-25 }
  }
  |\csvautotabular| or |\csvautotabular*|
  is an abbreviation for the application of the option key
  \refKey{csvsim/autotabular} or \refKey{csvsim/autotabular*}
  together with other \meta{options} to \refCom{csvloop}.
  This macro reads the whole CSV file denoted by \meta{file name}
  with an automated formatting.
  The star variant treats the first line as data line and not as header line.
\begin{dispExample}
\csvautotabular*{grade.csv}
\end{dispExample}
\begin{dispExample}
\csvautotabular[filter equal={\csvcoliv}{f}]{grade.csv}
\end{dispExample}
\end{docCommands}



\begin{docCommands}[
    doc parameter = \oarg{options}\marg{file name}
  ]
  {
    { doc name = csvautolongtable  },
    { doc name = csvautolongtable*, doc new = 2021-06-25 }
  }
  |\csvautolongtable| or |\csvautolongtable*|
  is an abbreviation for the application of the option key
  \refKey{csvsim/autolongtable} or \refKey{csvsim/autolongtable*}
  together with other \meta{options} to \refCom{csvloop}.
  This macro reads the whole CSV file denoted by \meta{file name}
  with an automated formatting.
  For application, the package \ctanpkg{longtable} is required which has to be
  loaded in the preamble.
  The star variant treats the first line as data line and not as header line.
\begin{dispListing}
\csvautolongtable{grade.csv}
\end{dispListing}
\csvautolongtable{grade.csv}
\end{docCommands}


\clearpage


\begin{docCommands}[
    doc parameter = \oarg{options}\marg{file name}
  ]
  {
    { doc name = csvautobooktabular  },
    { doc name = csvautobooktabular*, doc new = 2021-06-25 }
  }
  |\csvautobooktabular| or |\csvautobooktabular*|
  is an abbreviation for the application of the option key
  \refKey{csvsim/autobooktabular} or \refKey{csvsim/autobooktabular*}
  together with other \meta{options} to \refCom{csvloop}.
  This macro reads the whole CSV file denoted by \meta{file name}
  with an automated formatting.
  For application, the package \ctanpkg{booktabs} is required which has to be
  loaded in the preamble.
  The star variant treats the first line as data line and not as header line.
\begin{dispExample}
\csvautobooktabular{grade.csv}
\end{dispExample}
\end{docCommands}


\begin{docCommands}[
    doc parameter = \oarg{options}\marg{file name}
  ]
  {
    { doc name = csvautobooklongtable  },
    { doc name = csvautobooklongtable*, doc new = 2021-06-25 }
  }
  |\csvautobooklongtable| or |\csvautobooklongtable*|
  is an abbreviation for the application of the option key
  \refKey{csvsim/autobooklongtable} or \refKey{csvsim/autobooklongtable*}
  together with other \meta{options} to \refCom{csvloop}.
  This macro reads the whole CSV file denoted by \meta{file name}
  with an automated formatting.
  For application, the packages \ctanpkg{booktabs} and \ctanpkg{longtable} are required which have to be
  loaded in the preamble.
  The star variant treats the first line as data line and not as header line.
\begin{dispListing}
\csvautobooklongtable{grade.csv}
\end{dispListing}
\csvautobooklongtable{grade.csv}
\end{docCommands}


\clearpage

\begin{docCommands}[
    doc parameter = \oarg{options}\marg{file name}\oarg{taboptions 1}\oarg{taboptions 2},
  ]
  {
    { doc name = csvautotabularray, doc new and updated={2023-10-13}{2023-10-17}, },
    { doc name = csvautotabularray* },
    { doc name = csvautolongtabularray  },
    { doc name = csvautolongtabularray* },
  }
  These macros are abbreviations for the application of the option keys
  \refKey{csvsim/autotabularray}, \refKey{csvsim/autotabularray*},\\
  \refKey{csvsim/autolongtabularray}, or \refKey{csvsim/autolongtabularray*}
  together with other \meta{options} to \refCom{csvloop}.
  These macros read the whole CSV file denoted by \meta{file name}
  with an automated formatting.
  For application, the package \ctanpkg{tabularray} is required which has to be
  loaded in the preamble.
  \refCom{csvautotabularray} uses the \docAuxEnvironment*{tblr} environment and
  \refCom{csvautolongtabularray} uses the \docAuxEnvironment*{longtblr} environment.
  The star variants treat the first line as data line and not as header line.\par
  Options to the table environments from \ctanpkg{tabularray} may be given
  by either setting \refKey{csvsim/generic table options} or
  using \meta{taboptions 1} and \meta{taboptions 2}.\par
  The default setting is
\begin{dispListing}
generic table options =
  { {
    row{1}     = {font=\bfseries,preto=\MakeUppercase},
    hline{1,Z} = {0.08em},
    hline{2}   = {0.05em},
  } }
\end{dispListing}
  For the star variants, the default setting is
\begin{dispListing}
generic table options =
  { {
    hline{1,Z} = {0.08em},
  } }
\end{dispListing}

Examples:

\begin{dispExample}
\csvautotabularray{grade.csv}
\end{dispExample}

\begin{dispExample}
\csvautotabularray[table centered,
    generic table options =
    {{
      row{odd}   = {red!85!gray!7},
      row{1}     = {bg=red!85!gray, fg=white,
                    font=\bfseries, preto=\MakeUppercase},
    }}
  ] {grade.csv}
\end{dispExample}

\clearpage

Alternatively to \refKey{csvsim/generic table options}
(and overruling this option), one may give options to
\docAuxEnvironment*{tblr} or \docAuxEnvironment*{longtblr}
using \meta{taboptions 1} and \meta{taboptions 2}.
If \meta{taboptions 2} is \emph{not present}, then \meta{taboptions 1}
is used as
mandatory argument (\ctanpkg{tabularray} inner specification).
Otherwise, \meta{taboptions 1} is used as
optional argument (\ctanpkg{tabularray} outer specification)
and
\meta{taboptions 2} as
mandatory argument (\ctanpkg{tabularray} inner specification).

\begin{dispExample}
\csvautotabularray[table centered]
  {grade.csv}
  [
    row{odd}   = {red!85!gray!7},
    row{1}     = {bg=red!85!gray, fg=white,
                  font=\bfseries, preto=\MakeUppercase},
  ]
\end{dispExample}


\begin{dispExample}
\csvautotabularray[table centered]
  {grade.csv}
  [
    tall,
    caption      = {My table},
    remark{Note} = {My remark},
  ]
  [
    row{odd}   = {red!85!gray!7},
    row{1}     = {bg=red!85!gray, fg=white,
                  font=\bfseries, preto=\MakeUppercase},
  ]
\end{dispExample}


\end{docCommands}


\clearpage
\section{Option Keys}\label{sec:schluessel}%
For the \meta{options} in \refCom{csvreader} respectively \refCom{csvloop}
the following |l3keys| keys can be applied. The \meta{module} name |csvsim| is not
to be used inside these macros.


\subsection{Command Definition}%--------%[[

\begin{docCsvKey}{before reading}{=\meta{code}}{no default, initially empty}
  Sets the \meta{code} to be executed before the CSV file is opened.
\end{docCsvKey}

\begin{docCsvKey}{after head}{=\meta{code}}{no default, initially empty}
  Sets the \meta{code} to be executed after the header line is read.
  \refCom{thecsvcolumncount} and header entries are available.
\end{docCsvKey}

\begin{docCsvKey}{before filter}{=\meta{code}}{no default, initially empty}
  Sets the \meta{code} to be executed after reading and consistency checking
  of a data line. It is executed before any filter condition is checked,
  see e.g. \refKey{csvsim/filter ifthen} and
  also see \refKey{csvsim/full filter}.
  No additions to the input stream should be given here.
  All line entries are available.
\end{docCsvKey}

\begin{docCsvKey}[][doc new=2021-07-06]{after filter}{=\meta{code}}{no default, initially empty}
  Sets the \meta{code} to be executed for an accepted line after
  \refKey{csvsim/late after line} and before \refKey{csvsim/before line}.
  All line entries are available.
  No additions to the input stream should be given here. \meta{code} may
  contain processing of data content to generate new values.
\end{docCsvKey}

\begin{docCsvKey}{late after head}{=\meta{code}}{no default, initially empty}
  Sets the \meta{code} to be executed after reading and disassembling
  of the first accepted data line.
  These operations are executed before further processing of this line.
  \meta{code} should not refer to any data content, but may be something
  like |\\|.
\end{docCsvKey}

\begin{docCsvKey}{late after line}{=\meta{code}}{no default, initially empty}
  Sets the \meta{code} to be executed after reading and disassembling
  of the next accepted data line (after \refKey{csvsim/before filter}).
  These operations are executed before further processing of this line.
  \meta{code} should not refer to any data content, but may be something
  like |\\|.
  \refKey{csvsim/late after line} overwrites
  \refKey{csvsim/late after first line} and
  \refKey{csvsim/late after last line}.
  Note that table options like \refKey{csvsim/tabular} set this key to |\\|
  automatically.
\end{docCsvKey}


\begin{docCsvKey}{late after first line}{=\meta{code}}{no default, initially empty}
  Sets the \meta{code} to be executed after reading and disassembling
  of the second accepted data line instead of \refKey{csvsim/late after line}.
  \meta{code} should not refer to any data content.
  This key has to be set after \refKey{csvsim/late after line}.
\end{docCsvKey}


\begin{docCsvKey}{late after last line}{=\meta{code}}{no default, initially empty}
  Sets the \meta{code} to be executed after processing of the last
  accepted data line instead of \refKey{csvsim/late after line}.
  \meta{code} should not refer to any data content.
  This key has to be set after \refKey{csvsim/late after line}.
\end{docCsvKey}


\begin{docCsvKey}{before line}{=\meta{code}}{no default, initially empty}
  Sets the \meta{code} to be executed after \refKey{csvsim/after filter}
  and before \refKey{csvsim/command}.
  All line entries are available.
  \refKey{csvsim/before line} overwrites
  \refKey{csvsim/before first line}.
\end{docCsvKey}


\begin{docCsvKey}{before first line}{=\meta{code}}{no default, initially empty}
  Sets the \meta{code} to be executed instead of \refKey{csvsim/before line}
  for the first accepted data line.
  All line entries are available.
  This key has to be set after \refKey{csvsim/before line}.
\end{docCsvKey}

\pagebreak

\begin{docCsvKey}{command}{=\meta{code}}{no default, initially \cs{csvline}}
  Sets the \meta{code} to be executed for every accepted data line.
  It is executed between \refKey{csvsim/before line} and \refKey{csvsim/after line}.
  \refKey{csvsim/command} describes the main processing of the line
  entries. \refCom{csvreader} sets \refKey{csvsim/command} as mandatory
  parameter.
\end{docCsvKey}

\begin{docCsvKey}{after line}{=\meta{code}}{no default, initially empty}
  Sets the \meta{code} to be executed for every accepted data line
  after \refKey{csvsim/command}.
  All line entries are still available.
  \refKey{csvsim/after line} overwrites \refKey{csvsim/after first line}.
\end{docCsvKey}


\begin{docCsvKey}{after first line}{=\meta{code}}{no default, initially empty}
  Sets the \meta{code} to be executed instead of \refKey{csvsim/after line}
  for the first accepted data line.
  All line entries are still available.
  This key has to be set after \refKey{csvsim/after line}.
\end{docCsvKey}

\begin{docCsvKey}{after reading}{=\meta{code}}{no default, initially empty}
  Sets the \meta{code} to be executed after the CSV file is closed.
\end{docCsvKey}

\bigskip

The following example illustrates the sequence of command execution.
Note that \refKey{csvsim/command} is set by the mandatory last
parameter of \refCom{csvreader}.

\begin{dispExample}
\csvreader[
  before reading        = \meta{before reading}\\,
  after head            = \meta{after head},
  before filter         = \\\meta{before filter},
  after filter          = \meta{after filter},
  late after head       = \meta{late after head},
  late after line       = \meta{late after line},
  late after first line = \meta{late after first line},
  late after last line  = \\\meta{late after last line},
  before line           = \meta{before line},
  before first line     = \meta{before first line},
  after line            = \meta{after line},
  after first line      = \meta{after first line},
  after reading         = \\\meta{after reading}
    ]{grade.csv}{name=\name}{\textbf{\name}}%
\end{dispExample}

Additional command definition keys are provided for the supported tables,
see Section~\ref{subsec:tabsupport} from page~\pageref{subsec:tabsupport}.

\clearpage
\subsection{Header Processing and Column Name Assignment}%

\begin{docCsvKey}[doc updated=2024-09-18]{head}{\colOpt{=true\textbar false}}{default |true|, initially |true|}
  If this key is set, the first non-empty line of the CSV file is treated as a header
  line which can be used for column name assignments.
\end{docCsvKey}


\begin{docCsvKey}{no head}{}{no value}
  Abbreviation for |head=false|, i.\,e. the first non-empty line of the CSV file is
  treated as data line.
  Note that this option cannot be used in combination with
  the |\csvauto...| commands like \refCom{csvautotabular}, etc.
  Instead, there are \emph{star} variants like \refCom{csvautotabular*} to
  process files without header line.
  See Section~\ref{noheader} on page~\pageref{noheader} for examples.
\end{docCsvKey}

\begin{docCsvKey}{column names}{=\marg{assignments}}{no default, initially empty}
  Adds some new \meta{assignments} of macros to columns in key value syntax.
  Existing assignments are kept.\par
  The \meta{assignments} are given as comma separated list of key value pairs
  \mbox{\meta{name}|=|\meta{macro}}. Here, \meta{name} is an entry from the
  header line \emph{or} the arabic number of the addressed column.
  \meta{macro} is some \TeX\ macro which gets the content of the addressed column.
\begin{dispListing}
  column names = {name=\surname, givenname=\firstname, grade=\grade}
\end{dispListing}
\end{docCsvKey}


\begin{docCsvKey}{column names reset}{}{no value}
  Clears all assignments of macros to columns.
\end{docCsvKey}


\begin{docCsvKey}{head to column names}{\colOpt{=true\textbar false}}{default |true|, initially |false|}
  If this key is set, the entries of the header line are used automatically
  as macro names for the columns. This option can be used only, if
  the header entries do not contain spaces and special characters to be
  used as feasible \LaTeX\ macro names.
  Note that the macro definition is \emph{global} and may therefore override
  existing macros for the rest of the document. Adding
  \refKey{csvsim/head to column names prefix} may help to avoid unwanted
  overrides.
\end{docCsvKey}


\begin{docCsvKey}[][doc new=2019-07-16]{head to column names prefix}{=\meta{text}}{no default, initially empty}
  The given \meta{text} is prefixed to the name of all macros generated by
  \refKey{csvsim/head to column names}. For example, if you use the settings
\begin{dispListing}
    head to column names,
    head to column names prefix=MY,
\end{dispListing}
  a header entry |section| will generate the corresponding macro
  |\MYsection| instead of destroying the standard \LaTeX\ |\section| macro.
\end{docCsvKey}


\begin{docCsvKey}[][doc new=2022-02-01]{column names detection}{\colOpt{=true\textbar false}}{default |true|, initially |true|}
  If this key is set, the header line is detected for names which can be used
  for \refKey{csvsim/column names} and \refKey{csvsim/head to column names}.
  Otherwise, these options are not functional.\\
  This key can and should be set to |false|, if the header line contains
  macros or characters not allowed inside \LaTeX\ control sequences, because
  otherwise compilation error are to be expected.
\end{docCsvKey}


\clearpage
\subsection{Consistency Check}\label{sec:consistency}%

\begin{docCsvKey}{check column count}{\colOpt{=true\textbar false}}{default |true|, initially |true|}
  This key defines, whether the number of entries in a data line is checked against
  an expected value or not.\\
  If |true|, every non consistent line is ignored without announcement.\\
  If |false|, every line is accepted and may produce an error during
  further processing.
\end{docCsvKey}


\begin{docCsvKey}{no check column count}{}{no value}
  Abbreviation for |check column count=false|.
\end{docCsvKey}


\begin{docCsvKey}[][doc updated=2021-06-24]{column count}{=\meta{number}}{no default, initially |0|}
  Sets the \meta{number} of feasible entries per data line.
  If \refKey{csvsim/column count} is set to |0|, the number of entries of
  the first non-empty line determines the column count (automatic detection).

  This setting is only useful in connection with \refKey{csvsim/no head},
  since \meta{number} would be replaced by the number of entries in the
  header line otherwise.
\end{docCsvKey}


\begin{docCsvKey}{on column count error}{=\meta{code}}{no default, initially empty}
  \meta{code} to be executed for unfeasible data lines.
\end{docCsvKey}


\begin{docCsvKey}{warn on column count error}{}{style, no value}
  Display of a warning for unfeasible data lines.
\end{docCsvKey}


\clearpage
\subsection{Filtering}\label{subsec:filtering}%

Applying a \emph{filter} means that data lines are only processed / displayed,
if they fulfill a given \emph{condition}.

The following string compare filters \refKey{csvsim/filter strcmp} and
\refKey{csvsim/filter equal} are identical by logic, but differ in implementation.

\begin{docCsvKey}[][doc updated=2022-10-21]{filter strcmp}{=\marg{stringA}\marg{stringB}}{no default}
  Only lines where \meta{stringA} and \meta{stringB} are equal after expansion
  are accepted.
  The implementation is done with \docAuxCommand*{str_if_eq_p:ee}.
\begin{dispExample}
% \usepackage{booktabs}
\csvreader[
    head to column names,
    tabular    = llll,
    table head = \toprule & \bfseries Name & \bfseries Matr & \bfseries Grade\\\midrule,
    table foot = \bottomrule,
    filter strcmp = {\gender}{f},  %>> list only female persons <<
  ]{grade.csv}{}{%
    \thecsvrow & \slshape\name, \givenname & \matriculation & \grade
  }
\end{dispExample}
\end{docCsvKey}


\begin{docCsvKey}[][doc updated=2022-10-21]{filter not strcmp}{=\marg{stringA}\marg{stringB}}{no default}
  Only lines where \meta{stringA} and \meta{stringB} are not equal after expansion
  are accepted.
  The implementation is done with \docAuxCommand*{str_if_eq_p:ee}.
\end{docCsvKey}


\begin{docCsvKey}{filter equal}{=\marg{stringA}\marg{stringB}}{no default}
  Only lines where \meta{stringA} and \meta{stringB} are equal after expansion
  are accepted.
  The implementation is done with the \ctanpkg{ifthen} package (loading required!).
\end{docCsvKey}


\begin{docCsvKey}{filter not equal}{=\marg{stringA}\marg{stringB}}{no default}
  Only lines where \meta{stringA} and \meta{stringB} are not equal after expansion
  are accepted.
  The implementation is done with the \ctanpkg{ifthen} package (loading required!).
\end{docCsvKey}


\enlargethispage*{1cm}
\begin{docCsvKey}[][doc new and updated={2021-06-25}{2022-10-21}]{filter fp}{=\marg{floating point comparison}}{no default}
  Only data lines which fulfill a \LaTeX3 \meta{floating point comparison}
  are accepted. The evaluation is done using \docAuxCommand*{fp_compare_p:n}.
\begin{dispExample}
% \usepackage{booktabs}
\csvreader[
    head to column names,
    tabular    = llll,
    table head = \toprule & \bfseries Name & \bfseries Matr & \bfseries Grade\\\midrule,
    table foot = \bottomrule,
  %>> list only matriculation numbers greater than 20000
  %   and grade less than 4.0 <<
    filter fp  = {  \matriculation > 20000  &&  \grade < 4.0  },
  ]{grade.csv}{}{%
    \thecsvrow & \slshape\name, \givenname & \matriculation & \grade
  }
\end{dispExample}
\end{docCsvKey}


\clearpage

\begin{docCsvKey}[][doc new and updated={2021-06-25}{2022-10-21}]{filter bool}{=\marg{boolean expression}}{no default}
  Only data lines which fulfill a \LaTeX3 \meta{boolean expression} are accepted.
  Note that such an \meta{boolean expression} needs expl3 code.
  To preprocess the data line before testing the \meta{boolean expression},
  the option key \refKey{csvsim/before filter} can be used.
\begin{dispExample}
% For convenience, we save the filter
\ExplSyntaxOn
%>> list only matriculation numbers greater than 20000, list only men <<
\csvstyle{myfilter}
  {
    filter~bool =
      {
        \int_compare_p:n { \matriculation > 20000 } &&
        \str_if_eq_p:ee  { \gender }{ m }
      }
  }
\ExplSyntaxOff

\csvreader[
    head to column names,
    tabular    = llll,
    table head = \toprule & \bfseries Name & \bfseries Matr & \bfseries Grade\\\midrule,
    table foot = \bottomrule,
    myfilter
  ]{grade.csv}{}{%
    \thecsvrow & \slshape\name, \givenname & \matriculation & \grade
  }
\end{dispExample}
\end{docCsvKey}

\medskip
\begin{docCommand}[doc new=2021-06-25]{csvfilterbool}{\marg{key}\marg{boolean expression}}
  Defines a new |l3keys| meta key which applies \refKey{csvsim/filter bool}
  with the given \meta{boolean expression}.
\begin{dispExample}
% For convenience, we save the filter
\ExplSyntaxOn
%>> list only matriculation numbers greater than 20000, list only men <<
\csvfilterbool{myfilter}
  {
    \int_compare_p:n { \matriculation > 20000 } &&
    \str_if_eq_p:ee  { \gender }{ m }
  }
\ExplSyntaxOff

\csvreader[
    head to column names,
    tabular    = llll,
    table head = \toprule & \bfseries Name & \bfseries Matr & \bfseries Grade\\\midrule,
    table foot = \bottomrule,
    myfilter
  ]{grade.csv}{}{%
    \thecsvrow & \slshape\name, \givenname & \matriculation & \grade
  }
\end{dispExample}
\end{docCommand}


\clearpage

The following filter options are \emph{appendable} to the expl3 based
filter options:
\begin{itemize}
\item \refKey{csvsim/filter strcmp}
\item \refKey{csvsim/filter not strcmp}
\item \refKey{csvsim/filter fp}
\item \refKey{csvsim/filter bool}
\end{itemize}

\begin{docCsvKeys}[
    doc parameter   = {=\marg{stringA}\marg{stringB}},
    doc description = {no default},
    doc new = {2022-10-21}
  ]
  {
    { doc name = and filter strcmp  },
    { doc name = or filter strcmp },
  }
  Like \refKey{csvsim/filter strcmp}, but appended to a required existing
  expl3 based filter with \emph{and} (|&&|) resp. \emph{or} (\texttt{\textbar\textbar}).

\begin{dispExample}
\csvreader[
    head to column names,
    tabular    = llll,
    table head = \toprule & \bfseries Name & \bfseries Matr & \bfseries Grade\\\midrule,
    table foot = \bottomrule,
    filter fp  = {\matriculation>20000},
    and filter strcmp = {\gender}{m},
  ]{grade.csv}{}{%
    \thecsvrow & \slshape\name, \givenname & \matriculation & \grade
  }
\end{dispExample}
\end{docCsvKeys}


\begin{docCsvKeys}[
    doc parameter   = {=\marg{stringA}\marg{stringB}},
    doc description = {no default},
    doc new = {2022-10-21}
  ]
  {
    { doc name = and filter not strcmp  },
    { doc name = or filter not strcmp },
  }
  Like \refKey{csvsim/filter not strcmp}, but appended to a required existing
  expl3 based filter with \emph{and} (|&&|) resp. \emph{or} (\texttt{\textbar\textbar}).
\end{docCsvKeys}


\begin{docCsvKeys}[
    doc parameter   = {=\marg{floating point comparison}},
    doc description = {style, no default},
    doc new = {2022-10-21}
  ]
  {
    { doc name = and filter fp  },
    { doc name = or filter fp },
  }
  Like \refKey{csvsim/filter fp}, but appended to a required existing
  expl3 based filter with \emph{and} (|&&|) resp. \emph{or} (\texttt{\textbar\textbar}).
\begin{dispExample}
% \usepackage{booktabs}
\csvreader[
    head to column names,
    tabular    = llll,
    table head = \toprule & \bfseries Name & \bfseries Matr & \bfseries Grade\\\midrule,
    table foot = \bottomrule,
  %>> list only matriculation numbers greater than 20000 and grade less than 4.0 <<
    filter fp      = { \matriculation > 20000 },
    and filter fp  = { \grade < 4.0  },
  ]{grade.csv}{}{%
    \thecsvrow & \slshape\name, \givenname & \matriculation & \grade
  }
\end{dispExample}
\end{docCsvKeys}

\enlargethispage*{1cm}
\begin{docCsvKeys}[
    doc parameter   = {=\marg{boolean expression}},
    doc description = {style, no default},
    doc new = {2022-10-21}
  ]
  {
    { doc name = and filter bool },
    { doc name = or filter bool },
  }
  Like \refKey{csvsim/filter bool}, but appended to a required existing
  expl3 based filter with \emph{and} (|&&|) resp. \emph{or} (\texttt{\textbar\textbar}).
\end{docCsvKeys}


\clearpage

\begin{docCsvKey}[][doc new=2016-07-01]{filter test}{=\meta{condition}}{no default}
  Only data lines which fulfill a logical \meta{condition} are accepted.
  For the \meta{condition}, every single test normally employed like
\begin{dispListing}
\iftest{some testing}{true}{false}
\end{dispListing}
  can be used as
\begin{dispListing}
filter test=\iftest{some testing},
\end{dispListing}
  For |\iftest|, tests from the \ctanpkg{etoolbox} package like
  |\ifnumcomp|, |\ifdimgreater|, etc. and from \Fullref{sec:stringtests} can be used.
  Also, arbitrary own macros fulfilling this signature can be applied.
\begin{dispExample}
% \usepackage{etoolbox,booktabs}
\csvreader[
    head to column names,
    tabular    = llll,
    table head = \toprule & \bfseries Name & \bfseries Matr & \bfseries Grade\\\midrule,
    table foot = \bottomrule,
    %>> list only matriculation numbers greater than 20000 <<
    filter test = \ifnumgreater{\matriculation}{20000},
  ]{grade.csv}{}{%
    \thecsvrow & \slshape\name, \givenname & \matriculation & \grade
  }
\end{dispExample}
\end{docCsvKey}


\medskip
\begin{docCsvKey}[][doc new=2016-07-01]{filter expr}{=\meta{boolean expression}}{no default}
  Only data lines which fulfill a \meta{boolean expression} are accepted.
  Every \meta{boolean expression}
  from the \ctanpkg{etoolbox} package is feasible (package loading required!).
  To preprocess the data line before testing the \meta{boolean expression},
  the option key \refKey{csvsim/before filter} can be used.
\begin{dispExample}
% \usepackage{etoolbox,booktabs}
\csvreader[
    head to column names,
    tabular    = llll,
    table head = \toprule & \bfseries Name & \bfseries Matr & \bfseries Grade\\\midrule,
    table foot = \bottomrule,
    %>> list only matriculation numbers greater than 20000
    %   and grade less than 4.0 <<
    filter expr = {     test{\ifnumgreater{\matriculation}{20000}}
                    and test{\ifdimless{\grade pt}{4.0pt}}          },
  ]{grade.csv}{}{%
    \thecsvrow & \slshape\name, \givenname & \matriculation & \grade
  }
\end{dispExample}
\end{docCsvKey}


\clearpage
\begin{docCsvKey}[][doc new=2016-07-01]{filter ifthen}{=\meta{boolean expression}}{no default}
  Only data lines which fulfill a \meta{boolean expression} are accepted.
  For the \meta{boolean expression}, every term from the \ctanpkg{ifthen} package
  is feasible (package loading required!).
  To preprocess the data line before testing the \meta{boolean expression},
  the option key \refKey{csvsim/before filter} can be used.

\begin{dispExample}
% \usepackage{ifthen,booktabs}
\csvreader[
    head to column names,
    tabular    = llll,
    table head = \toprule & \bfseries Name & \bfseries Matr & \bfseries Grade\\\midrule,
    table foot = \bottomrule,
    %>> list only female persons <<
    filter ifthen=\equal{\gender}{f},
  ]{grade.csv}{}{%
    \thecsvrow & \slshape\name, \givenname & \matriculation & \grade
  }
\end{dispExample}
\end{docCsvKey}


\begin{docCsvKey}{no filter}{}{no value, initially set}
  Clears a set filter.
\end{docCsvKey}


\begin{docCsvKey}{filter accept all}{}{no value, initially set}
  Alias for |no filter|. All consistent data lines are accepted.
\end{docCsvKey}


\begin{docCsvKey}{filter reject all}{}{no value}
  All data line are ignored.
\end{docCsvKey}


\begin{docCsvKey}[][doc new=2016-07-01]{full filter}{=\meta{code}}{no default}
  Technically, this key is an alias for \refKey{csvsim/before filter}.
  Philosophically, \refKey{csvsim/before filter} computes something before
  a filter condition is set, but \refKey{csvsim/full filter} should implement
  the full filtering. Especially, \refCom{csvfilteraccept} or
  \refCom{csvfilterreject} \emph{should} be set inside the \meta{code}.
\begin{dispExample}
% \usepackage{etoolbox,booktabs}
\csvreader[
    head to column names,
    tabular    = llll,
    table head = \toprule & \bfseries Name & \bfseries Matr & \bfseries Grade\\\midrule,
    table foot = \bottomrule,
    %>> list only matriculation numbers greater than 20000
    %   and grade less than 4.0 <<
    full filter = \ifnumgreater{\matriculation}{20000}
                  {\ifdimless{\grade pt}{4.0pt}{\csvfilteraccept}{\csvfilterreject}}
                  {\csvfilterreject},
  ]{grade.csv}{}{%
    \thecsvrow & \slshape\name, \givenname & \matriculation & \grade
  }
\end{dispExample}
\end{docCsvKey}



%]]

\clearpage
\subsection{Line Range}\label{subsec:linerange}

Applying a \emph{line range} means to select certain line numbers to be
displayed. These line numbers are not necessarily line numbers of
the input file, see \refCom{thecsvinputline}, but line numbers of
type \refCom{thecsvrow}.

For example, if a \emph{filter} was applied, see \Fullref{subsec:filtering},
and 42 lines are accepted, a \emph{range} could select the first 20 of them or
line 10 to 30 of the accepted lines.


\begin{docCsvKey}[][doc new and updated={2021-06-29}{2022-09-21}]{range}{=\brackets{\meta{range1},\meta{range2},\meta{range3},... }}{no default, initially empty}
  Defines a comma separated list of line ranges. If a line number \refCom{thecsvrow}
  satisfies one or more of the given \meta{range1}, \meta{range2}, \ldots,
  the corresponding line is processed and displayed.
  If \refKey{csvsim/range} is set to empty, all lines are accepted.

  Every \meta{range} can
  corresponds to one of the following variants:
  \begin{tabbing}
  \hspace*{2cm}\=\kill
  \texttt{\meta{a}-\meta{b}} \> meaning line numbers \meta{a} to \meta{b}.\\
  \texttt{\meta{a}-}         \> meaning line numbers \meta{a} to |\c_max_int|=2 147 483 647.\\
  \texttt{-\meta{b}}         \> meaning line numbers 1 to \meta{b}.\\
  \texttt{-}                 \> meaning line numbers 1 to 2 147 483 647 (inefficient; don't use).\\
  \texttt{\meta{a}}          \> meaning line numbers \meta{a} to \meta{a} (i.e. only \meta{a}).\\
  \texttt{\meta{a}+\meta{d}} \> meaning line numbers \meta{a} to \meta{a}$+$\meta{d}$-1$.\\
  \texttt{\meta{a}+}         \> meaning line numbers \meta{a} to \meta{a} (i.e. only \meta{a}).\\
  \texttt{+\meta{d}}         \> meaning line numbers 1 to \meta{d}.\\
  \texttt{+}                 \> meaning line numbers 1 to 1 (i.e. only 1; weird).\\
  \end{tabbing}

\begin{dispExample}
% \usepackage{booktabs}
\csvreader[
    head to column names,
    range      = 2-3,
    tabular    = llll,
    table head = \toprule & \bfseries Name & \bfseries Matr & \bfseries Grade\\\midrule,
    table foot = \bottomrule,
  ]{grade.csv}{}{%
    \thecsvrow & \slshape\name, \givenname & \matriculation & \grade
  }
\end{dispExample}


\begin{dispExample}
% \usepackage{booktabs}
\csvreader[
    head to column names,
    range      = 3-,
    tabular    = llll,
    table head = \toprule & \bfseries Name & \bfseries Matr & \bfseries Grade\\\midrule,
    table foot = \bottomrule,
  ]{grade.csv}{}{%
    \thecsvrow & \slshape\name, \givenname & \matriculation & \grade
  }
\end{dispExample}


\begin{dispExample}
% \usepackage{booktabs}
\csvreader[
    head to column names,
    range      = 2+2,
    tabular    = llll,
    table head = \toprule & \bfseries Name & \bfseries Matr & \bfseries Grade\\\midrule,
    table foot = \bottomrule,
  ]{grade.csv}{}{%
    \thecsvrow & \slshape\name, \givenname & \matriculation & \grade
  }
\end{dispExample}

\begin{dispExample}
% \usepackage{booktabs}
\csvreader[
    head to column names,
    range      = {2,4},
    tabular    = llll,
    table head = \toprule & \bfseries Name & \bfseries Matr & \bfseries Grade\\\midrule,
    table foot = \bottomrule,
  ]{grade.csv}{}{%
    \thecsvrow & \slshape\name, \givenname & \matriculation & \grade
  }
\end{dispExample}

To select the last $n$ lines, you have to know or count the line numbers first.
The following example displays the last three line numbers:

\begin{dispExample}
% \usepackage{booktabs}
\csvreader{grade.csv}{}{}%   count line numbers
\csvreader[
    head to column names,
    range      = {\thecsvrow-2}-,
    tabular    = llll,
    table head = \toprule & \bfseries Name & \bfseries Matr & \bfseries Grade\\\midrule,
    table foot = \bottomrule,
  ]{grade.csv}{}{%
    \thecsvrow & \slshape\name, \givenname & \matriculation & \grade
  }
\end{dispExample}

\end{docCsvKey}



\clearpage
\subsection{Table Support}\label{subsec:tabsupport}%--------%[[

\subsubsection{Predefined Tables}\label{subsubsec:table_predef}

\begin{docCsvKey}{tabular}{=\meta{table format}}{style, no default}
  Surrounds the CSV processing with |\begin{tabular}|\marg{table format}
  at begin and with |\end{tabular}| at end.
  Additionally, the commands defined by the key values of
  \refKey{csvsim/before table}, \refKey{csvsim/table head}, \refKey{csvsim/table foot},
  and \refKey{csvsim/after table} are executed at the appropriate places.
  \refKey{csvsim/late after line} is set to \cs{}\cs{}.
\end{docCsvKey}


\begin{docCsvKey}{centered tabular}{=\meta{table format}}{style, no default}
  Like \refKey{csvsim/tabular} but inside an additional |center| environment.
\end{docCsvKey}


\begin{docCsvKey}{longtable}{=\meta{table format}}{style, no default}
  Like \refKey{csvsim/tabular} but for the |longtable| environment.
  This requires the package \ctanpkg{longtable} (not loaded automatically).
\end{docCsvKey}


\begin{docCsvKey}{tabbing}{}{style, no value}
  Like \refKey{csvsim/tabular} but for the |tabbing| environment.
\end{docCsvKey}


\begin{docCsvKey}{centered tabbing}{}{style, no value}
  Like \refKey{csvsim/tabbing} but inside an additional |center| environment.
\end{docCsvKey}


\begin{docCsvKey}[][doc new=2021-07-06]{tabularray}{=\meta{table format}}{style, no default}
  Like \refKey{csvsim/tabular} but for the |tblr| environment.
  This requires the package \ctanpkg{tabularray} (not loaded automatically).
  This also sets \refKey{csvsim/collect data} since this kind of table
  needs collected content, see \Fullref{sec:datacollection}.
  Note that \refKey{csvsim/after reading} is set to use the collected
  data immediately. See \Fullref{sec:tabularray} for examples.
\end{docCsvKey}


\begin{docCsvKey}[][doc new=2021-07-23]{long tabularray}{=\meta{table format}}{style, no default}
  Like \refKey{csvsim/tabularray} but using the |longtblr| environment
  from the package \ctanpkg{tabularray} (not loaded automatically).
\end{docCsvKey}


\begin{docCsvKey}[][doc new=2021-07-06]{centered tabularray}{=\meta{table format}}{style, no default}
  Like \refKey{csvsim/tabularray} but inside an additional |center| environment.
\end{docCsvKey}


\begin{docCsvKey}{no table}{}{style, no value}
  Deactivates |tabular|-like environments activated by
  \refKey{csvsim/tabular}, \refKey{csvsim/longtable}, etc.
  Note that not all settings of \refKey{csvsim/tabularray} are reverted.
\end{docCsvKey}


\clearpage
\subsubsection{Additional Options for Tables}\label{subsubsec:table_options}

\begin{docCsvKey}{before table}{=\meta{code}}{no default, initially empty}
  Sets the \meta{code} to be executed before the begin of |tabular|-like environments,
  i.e. immediately before |\begin{tabular}|, etc.
\end{docCsvKey}


\begin{docCsvKey}{table head}{=\meta{code}}{no default, initially empty}
  Sets the \meta{code} to be executed after the begin of |tabular|-like environments,
  i.e. immediately after |\begin{tabular}|, etc.
\end{docCsvKey}


\begin{docCsvKey}{table foot}{=\meta{code}}{no default, initially empty}
  Sets the \meta{code} to be executed before the end of |tabular|-like environments,
  i.e. immediately before |\end{tabular}|, etc.
\end{docCsvKey}


\begin{docCsvKey}{after table}{=\meta{code}}{no default, initially empty}
  Sets the \meta{code} to be executed after the end of |tabular|-like environments,
  i.e. immediately after |\end{tabular}|, etc.
\end{docCsvKey}


\begin{docCsvKey}[][doc new=2021-09-09]{table centered}{\colOpt{=true\textbar false}}{default |true|, initially |false|}
  If |true|, the table is put inside an additional |center| environment.
  This environment begins before \refKey{csvsim/before table}
  and ends after \refKey{csvsim/after table}. The predefined |tabular|-like environments
  from Section~\fullref{subsubsec:table_predef} use this option internally,
  i.e. \mbox{|centered tabular={ccc}|} is identical to
  \mbox{|tabular={ccc}, table centered|}.
\end{docCsvKey}


\clearpage
\subsubsection{Generic Tables}\label{subsubsec:table_generic}
In Section~\fullref{subsubsec:table_predef}, several |tabular|-like environments
are described with predefined keys. The following keys allow to use further
|tabular|-like environments with configurable names and options.


\begin{docCsvKey}[][doc new=2021-09-09]{generic table}{=\meta{name}}{no default, initially empty}
  Surrounds the CSV processing with \cs{begin}\marg{name}
  at begin and with \cs{end}\marg{name} at end.
  Additionally, the commands defined by the key values of
  \refKey{csvsim/before table}, \refKey{csvsim/table head}, \refKey{csvsim/table foot},
  and \refKey{csvsim/after table} are executed at the appropriate places.
  \refKey{csvsim/late after line} is set to \cs{}\cs{}.\par
  If the environment \meta{name} takes options, these have to be set using
  \refKey{csvsim/generic table options}.

\begin{dispListing}
  % The `tabular` environment would be used like the following example
  ...
  generic table         = tabular,
  generic table options = {{ccllrr}},
  ...
\end{dispListing}
\end{docCsvKey}


\begin{docCsvKey}[][doc new and updated={2021-09-09}{2023-12-18}]{generic collected table}{=\meta{name}}{no default, initially empty}
  Like \refKey{csvsim/generic table} but for environments which need
  collected content, e.g. |tblr| from package \ctanpkg{tabularray}, see \Fullref{sec:datacollection}.
  Note that \refKey{csvsim/consume collected data} is set to |true| to
  use the collected data immediately.

\begin{dispListing}
  % The `tblr` environment from package `tabularray` would be used
  % like the following example
  ...
  generic collected table = tblr,
  generic table options   = {{rowsep=1mm, colsep=5mm}},
  ...
\end{dispListing}
\end{docCsvKey}


\begin{docCsvKey}[][doc new=2021-09-09]{generic table options}{=\marg{code}}{no default, initially empty}
  Places \meta{code} immediately after \cs{begin}\marg{name} set up with
  \refKey{csvsim/generic table} or \refKey{csvsim/generic collected table}.
  \meta{code} may contain any parameters the environment \meta{name} needs to have.
  \textbf{\color{red!50!black}You are strongly advised to use an extra pair of
  curly brackets \marg{code} around \meta{code}}, because the outer pair of braces is
  removed during option processing, see examples above.
\begin{dispListing}
  % Environment without parameters:
  generic table options =,
  % Environment with a mandatory parameter:
  generic table options = {{parameter}},
  % Environment with an optional and a mandatory parameter:
  generic table options = {[optional]{mandatory}},
  % Environment with two mandatory parameters:
  generic table options = {{mandatory 1}{mandatory 2}},
\end{dispListing}

\end{docCsvKey}


\clearpage
\subsubsection{General Survey Tables}\label{subsubsec:table_survey}

The following |auto| options are the counterparts for the respective quick
overview commands like \refCom{csvautotabular}, see Section~\ref{sec:autotable}.
They are listed for
completeness, but are unlikely to be used directly.

\begin{docCsvKeys}[
    doc parameter   = {=\meta{file name}},
    doc description = no default,
  ]
  {
    { doc name = autotabular, doc updated = {2022-02-01} },
    { doc name = autotabular* },
  }
  Reads the whole CSV file denoted \meta{file name} with an automated formatting.
  The star variant treats the first line as data line and not as header line.
\end{docCsvKeys}


\begin{docCsvKeys}[
    doc parameter   = {=\meta{file name}},
    doc description = no default,
  ]
  {
    { doc name = autolongtable, doc updated = {2022-02-01}  },
    { doc name = autolongtable* },
  }
  Reads the whole CSV file denoted \meta{file name} with an automated formatting
  using the required |longtable| package.
  The star variant treats the first line as data line and not as header line.
\end{docCsvKeys}


\begin{docCsvKeys}[
    doc parameter   = {=\meta{file name}},
    doc description = no default,
  ]
  {
    { doc name = autobooktabular, doc updated = {2022-02-01}  },
    { doc name = autobooktabular* },
  }
  Reads the whole CSV file denoted \meta{file name} with an automated formatting
  using the required |booktabs| package.
  The star variant treats the first line as data line and not as header line.
\end{docCsvKeys}


\begin{docCsvKeys}[
    doc parameter   = {=\meta{file name}},
    doc description = no default,
  ]
  {
    { doc name = autobooklongtable, doc updated = {2022-02-01}  },
    { doc name = autobooklongtable* },
  }
  Reads the whole CSV file denoted \meta{file name} with an automated formatting
  using the required |booktabs| and |longtable| packages.
  The star variant treats the first line as data line and not as header line.
\end{docCsvKeys}


\begin{docCsvKeys}[
    doc parameter   = {=\meta{file name}},
    doc description = no default,
    doc new = {2023-10-13}
  ]
  {
    { doc name = autotabularray },
    { doc name = autotabularray* },
    { doc name = autolongtabularray },
    { doc name = autolongtabularray* },
  }
  Reads the whole CSV file denoted \meta{file name} with an automated formatting
  using the required |tabularray| package.
  \refKey{csvsim/autotabularray} uses the \docAuxEnvironment*{tblr} environment and
  \refKey{csvsim/autolongtabularray} uses the \docAuxEnvironment*{longtblr} environment.
  The star variants treat the first line as data line and not as header line.
\end{docCsvKeys}


\clearpage
\subsection{Special Characters}\label{subsec:specchar}
Be default, the CSV content is treated like normal \LaTeX\ text, see
Subsection~\ref{macrocodexample} on page~\pageref{macrocodexample}.
For example, \%~can be used to start an in-line comment.
But, \TeX\ special characters of the CSV content may also be interpreted
as normal characters (|\catcode| 12, other), if one or more of the following options are used.

\begin{docCsvKey}{respect tab}{\colOpt{=true\textbar false}}{default |true|, initially |false|}
  If this key is set, every
  tabulator sign
  inside the CSV content is a normal character.
\end{docCsvKey}

\begin{docCsvKey}{respect percent}{\colOpt{=true\textbar false}}{default |true|, initially |false|}
  If this key is set, every
  percent sign \verbbox{\%}
  inside the CSV content is a normal character.
\end{docCsvKey}

\begin{docCsvKey}{respect sharp}{\colOpt{=true\textbar false}}{default |true|, initially |false|}
  If this key is set, every
  sharp sign \verbbox{\#}
  inside the CSV content is a normal character.
\end{docCsvKey}

\begin{docCsvKey}{respect dollar}{\colOpt{=true\textbar false}}{default |true|, initially |false|}
  If this key is set, every
  dollar sign \verbbox{\$}
  inside the CSV content is a normal character.
\end{docCsvKey}

\begin{docCsvKey}{respect and}{\colOpt{=true\textbar false}}{default |true|, initially |false|}
  If this key is set, every
  and sign \verbbox{\&}
  inside the CSV content is a normal character.
\end{docCsvKey}

\begin{docCsvKey}{respect backslash}{\colOpt{=true\textbar false}}{default |true|, initially |false|}
  If this key is set, every
  backslash sign \verbbox{\textbackslash}
  inside the CSV content is a normal character.
\end{docCsvKey}

\begin{docCsvKey}{respect underscore}{\colOpt{=true\textbar false}}{default |true|, initially |false|}
  If this key is set, every
  underscore sign \verbbox{\_}
  inside the CSV content is a normal character.
\end{docCsvKey}

\begin{docCsvKey}{respect tilde}{\colOpt{=true\textbar false}}{default |true|, initially |false|}
  If this key is set, every
  tilde sign \verbbox{\textasciitilde}
  inside the CSV content is a normal character.
\end{docCsvKey}

\begin{docCsvKey}{respect circumflex}{\colOpt{=true\textbar false}}{default |true|, initially |false|}
  If this key is set, every
  circumflex sign \verbbox{\textasciicircum}
  inside the CSV content is a normal character.
\end{docCsvKey}

\begin{docCsvKey}{respect leftbrace}{\colOpt{=true\textbar false}}{default |true|, initially |false|}
  If this key is set, every
  left brace sign \verbbox{\textbraceleft}
  inside the CSV content is a normal character.
\end{docCsvKey}

\begin{docCsvKey}{respect rightbrace}{\colOpt{=true\textbar false}}{default |true|, initially |false|}
  If this key is set, every
  right brace sign \verbbox{\textbraceright}
  inside the CSV content is a normal character.
\end{docCsvKey}

\begin{docCsvKey}{respect all}{}{style, no value, initially unset}
  Set all special characters from above to normal characters. This means
  a quite verbatim interpretation of the CSV content.
\end{docCsvKey}

\begin{docCsvKey}{respect none}{}{style, no value, initially set}
  Do not change any special character from above to normal character.
\end{docCsvKey}

\clearpage
\subsection{Separators}\label{sec:separators}%
\begin{docCsvKey}{separator}{=\meta{sign}}{no default, initially |comma|}
  \catcode `|=12
  Sets the \meta{sign} which is treated as separator between the data values
  of a data line. Feasible values are:
  \begin{itemize}
  \item\docValue{comma}: This is the initial value with '\texttt{,}' as separator.
  \medskip

  \item\docValue{semicolon}: Sets the separator to '\texttt{;}'.
\begin{dispExample}
% \usepackage{tcolorbox} for tcbverbatimwrite
\begin{tcbverbatimwrite}{testsemi.csv}
  name;givenname;matriculation;gender;grade
  Maier;Hans;12345;m;1.0
  Huber;Anna;23456;f;2.3
  Weißbäck;Werner;34567;m;5.0
\end{tcbverbatimwrite}

\csvautobooktabular[separator=semicolon]{testsemi.csv}
\end{dispExample}
\medskip

\item\docValue{pipe}: Sets the separator to '\texttt{|}'.
\begin{dispExample}
% \usepackage{tcolorbox} for tcbverbatimwrite
\begin{tcbverbatimwrite}{pipe.csv}
  name|givenname|matriculation|gender|grade
  Maier|Hans|12345|m|1.0
  Huber|Anna|23456|f|2.3
  Weißbäck|Werner|34567|m|5.0
\end{tcbverbatimwrite}

\csvautobooktabular[separator=pipe]{pipe.csv}
\end{dispExample}
\medskip

\item\docValue{tab}: Sets the separator to the tabulator sign.
  Automatically, \refKey{csvsim/respect tab} is set also.

\clearpage
\item\docValue{space}:\tcbdocmarginnote{\tcbdocnew{2023-05-08}}
  Sets the separator to space(s).
\begin{dispExample}
% \usepackage{tcolorbox} for tcbverbatimwrite
\begin{tcbverbatimwrite}{space.csv}
  name     givenname matriculation gender grade
  Maier    Hans      12345         m      1.0
  Huber    Anna      23456         f      2.3
  Weißbäck Werner    34567         m      5.0
\end{tcbverbatimwrite}

\csvautobooktabular[separator=space]{space.csv}
\end{dispExample}
  Note that leading spaces are ignored and multiple spaces are treated as one space.
  To denote an empty data cell insert \verb+{}+, e.g. \verb*+1 {} 3+.
  \end{itemize}
\end{docCsvKey}

\clearpage
\subsection{Miscellaneous}%

\begin{docCsvKey}{every csv}{}{style, initially empty}
  A meta key (style) definition which is used for every following CSV file.
  This definition can be overwritten with user code.
\begin{dispListing}
% Sets a warning message for unfeasible data lines.
\csvstyle{every csv}{warn on column count error}
\end{dispListing}
\end{docCsvKey}

\begin{docCsvKey}{default}{}{style}
  A style definition which is used for every following CSV file which
  resets all settings to default values\footnote{\texttt{default} is used
  because of the global nature of most settings.}.
  This key should not be used or changed by the user if there is not a
  really good reason (and you know what you do).
\end{docCsvKey}


\begin{docCsvKey}{file}{=\meta{file name}}{no default, initially |unknown.csv|}
  Sets the \meta{file name} of the CSV file to be processed.
  \refCom{csvreader} sets this option by a mandatory parameter.
\end{docCsvKey}


\begin{docCsvKey}{preprocessed file}{=\meta{file name}}{no default, initially \texttt{\textbackslash\detokenize{jobname_sorted.csv}}}
  Sets the \meta{file name} of the CSV file which is the output of a
  preprocessor.
\end{docCsvKey}


\begin{docCsvKey}{preprocessor}{=\meta{macro}}{no default}
  Defines a preprocessor for the given CSV file.
  The \meta{macro} has to have two mandatory arguments. The first argument
  is the original CSV file which is set by \refKey{csvsim/file}.
  The second argument is the preprocessed CSV file
  which is set by \refKey{csvsim/preprocessed file}.\par\smallskip
  Typically, the \meta{macro} may call an external program which preprocesses
  the original CSV file (e.\,g. sorting the file) and creates the
  preprocessed CSV file. The later file is used by \refCom{csvreader}
  or \refCom{csvloop}.
\begin{dispListing}
\newcommand{\mySortTool}[2]{%
  % call to an external program to sort file #1 with resulting file #2
}

\csvreader[%
    preprocessed file = \jobname_sorted.csv,
    preprocessor      = \mySortTool,
  ]{some.csv}{}{%
  % do something
}
\end{dispListing}
See Subsection~\ref{sec:Sorting} on page~\pageref{sec:Sorting} for a
concrete sorting preprocessing implemented with an external tool.
\end{docCsvKey}


\begin{docCsvKey}{no preprocessing}{}{style, no value, initially set}
  Clears any preprocessing, i.\,e. preprocessing is switched of.
\end{docCsvKey}



\clearpage
\subsection{Sorting}\label{sec:Sorting}%
\TeX/\LaTeX\ was not born under a sorting planet. |csvsimple-l3| provides no
sorting of data lines by \LaTeX-methods since sorting can be done much faster
and much better by external tools.

First, one should consider the appropriate \emph{place} for sorting:
\begin{itemize}
\item CSV files may be sorted by a tool \emph{before} the \LaTeX\ document is processed
  at all. If the CSV data is not likely to change, this is the most efficient method.
\item CSV files may be sorted by a tool every time before the \LaTeX\ document is compiled.
  This could be automated by a shell script or some processing tool like |arara|.
\item CSV files may be sorted on-the-fly by a tool during compilation of
  a \LaTeX\ document. This is the most elegant but not the most efficient way.
\end{itemize}

The first two methods are decoupled from anything concerning |csvsimple-l3|.
For the third method, the \refKey{csvsim/preprocessor} option is made for.
This allows to access an external tool for sorting.
\emph{Which tool} is your choice.

\csvsorter\ was written as a companion tool for |csvsimple|.
It is an open source Java command-line tool for sorting CSV files, available at\\
\url{https://T-F-S.github.io/csvsorter/}\quad or\quad
\url{https://github.com/T-F-S/csvsorter}

It can be
used for all three sorting approaches described above.
There is special support for on-the-fly sorting with \csvsorter\ using the
following options.

\begin{enumerate}\bfseries
\item To use the sorting options, you have to install \csvsorter\ before!
\item You have to give permission to call external tools during
  compilation, i.\,e.\ the command-line options for |latex| have to include
  |-shell-escape|.
\end{enumerate}

\bigskip

\begin{docCsvKey}{csvsorter command}{=\meta{system command}}{no default, initially |csvsorter|}
  The \meta{system command} specifies the system call for \csvsorter\ (without the options).
  If \csvsorter\ was completely installed following its documentation, there is
  nothing to change here. If the |csvsorter.jar| file is inside the same
  directory as the \LaTeX\ source file, you may configure:% preferrably inside the preamble:
\begin{dispListing}
\csvset{csvsorter command=java -jar csvsorter.jar}
\end{dispListing}
\end{docCsvKey}

\begin{docCsvKey}{csvsorter configpath}{=\meta{path}}{no default, initially |.|}
  Sorting with \csvsorter\ is done using XML configuration files. If these files
  are not stored inside the same directory as the \LaTeX\ source file, a
  \meta{path} to access them can be configured:
\begin{dispListing}
\csvset{csvsorter configpath=xmlfiles}
\end{dispListing}
  Here, the configuration files would be stored in a subdirectory named |xmlfiles|.
\end{docCsvKey}

\begin{docCsvKey}{csvsorter log}{=\meta{file name}}{no default, initially |csvsorter.log|}
  Sets the log file of \csvsorter\ to the given \meta{file name}.
\begin{dispListing}
\csvset{csvsorter log=outdir/csvsorter.log}
\end{dispListing}
  Here, the log file is written to a subdirectory named |outdir|.
\end{docCsvKey}

\clearpage
\begin{docCsvKey}{csvsorter token}{=\meta{file name}}{no default, initially |\textbackslash jobname.csvtoken|}
  Sets \meta{file name} as token file. This is an auxiliary file which
  communicates the success of \csvsorter\ to |csvsimple|.
\begin{dispListing}
\csvset{csvsorter log=outdir/\jobname.csvtoken}
\end{dispListing}
  Here, the token file is written to a subdirectory named |outdir|.
\end{docCsvKey}


\begin{docCsvKey}{sort by}{=\meta{file name}}{style, initially unset}
  The \meta{file name} denotes an XML configuration file for \csvsorter.
  Setting this option inside \refCom{csvreader} or
  \refCom{csvloop} will issue a system call to \csvsorter.
  \begin{itemize}
  \item \csvsorter\ uses the given CSV file as input file.
  \item \csvsorter\ uses \meta{file name} as configuration file.
  \item The output CSV file is denoted by \refKey{csvsim/preprocessed file}
     which is by default \texttt{\textbackslash\detokenize{jobname_sorted.csv}}.
     This output file is this actual file processed by \refCom{csvreader} or \refCom{csvloop}.
  \item \csvsorter\ also generates a log file denoted by \refKey{csvsim/csvsorter log} which is by default |csvsorter.log|.
  \end{itemize}

\par\medskip\textbf{First example:}
  To sort our example |grade.csv| file according to |name| and |givenname|, we
  use the following XML configuration file. Since \csvsorter\ uses double quotes
  as default brackets for column values, we remove bracket recognition to avoid
  a clash with the escaped umlauts of the example CSV file.\par\smallskip

\xmllisting{namesort}
\begin{dispExample}
% \usepackage{booktabs}
\csvreader[
    head to column names,
    sort by    = namesort.xml,
    tabular    = >{\color{red}}lllll,
    table head = \toprule Name & Given Name & Matriculation & Gender & Grade\\\midrule,
    table foot = \bottomrule
  ]{grade.csv}{}{%
    \csvlinetotablerow
  }
\end{dispExample}

\clearpage\textbf{Second example:}
  To sort our example |grade.csv| file according to |grade|, we
  use the following XML configuration file. Further, persons with the same |grade|
  are sorted by |name| and |givenname|. Since \csvsorter\ uses double quotes
  as default brackets for column values, we remove bracket recognition to avoid
  a clash with the escaped umlauts of the example CSV file.\par\smallskip

\xmllisting{gradesort}
\begin{dispExample}
% \usepackage{booktabs}
\csvreader[
    head to column names,
    sort by    = gradesort.xml,
    tabular    = llll>{\color{red}}l,
    table head = \toprule Name & Given Name & Matriculation & Gender & Grade\\\midrule,
    table foot = \bottomrule
  ]{grade.csv}{}{%
    \csvlinetotablerow
  }
\end{dispExample}

\clearpage\textbf{Third example:}
  To generate a matriculation/grade list, we sort our example |grade.csv| file
  using the following XML configuration file.
  Again, since \csvsorter\ uses double quotes
  as default brackets for column values, we remove bracket recognition to avoid
  a clash with the escaped umlauts of the example CSV file.\par\smallskip

\xmllisting{matriculationsort}
\begin{dispExample}
% \usepackage{booktabs}
\csvreader[
    head to column names,
    sort by    = matriculationsort.xml,
    tabular    = >{\color{red}}ll,
    table head = \toprule Matriculation & Grade\\\midrule,
    table foot = \bottomrule
  ]{grade.csv}{}{%
    \matriculation & \grade
  }
\end{dispExample}
\end{docCsvKey}


\clearpage
\begin{docCsvKey}{new sorting rule}{=\marg{name}\marg{file name}}{style, initially unset}
This is a convenience option to generate a new shortcut for often used
\refKey{csvsim/sort by} applications. It also adds a more semantic touch.
The new shortcut option is
\tcbox[on line,size=small,colback=white,colframe=red]{|sort by| \meta{name}} which expands to
\tcbox[on line,size=small,colback=white,colframe=red]{|sort by=|\marg{file name}}.\par\medskip

Consider the following example:
\begin{dispExample}
\csvautotabular[sort by=namesort.xml]{grade.csv}
\end{dispExample}
A good place for setting up a new sorting rule would be inside the preamble:

\csvset{new sorting rule={name}{namesort.xml}}
\begin{dispListing}
\csvset{new sorting rule={name}{namesort.xml}}
\end{dispListing}

Now, we can use the new rule:
\begin{dispExample}
\csvautotabular[sort by name]{grade.csv}
\end{dispExample}
\end{docCsvKey}


\begin{docCommand}[doc new=2021-06-28]{csvsortingrule}{\marg{name}\marg{file name}}
  Identical in function to \refKey{csvsim/new sorting rule}, see above.
A good place for setting up a new sorting rule would be inside the preamble:

\csvsortingrule{name}{namesort.xml}
\begin{dispListing}
\csvsortingrule{name}{namesort.xml}
\end{dispListing}

Now, we can use the new rule:
\begin{dispExample}
\csvautotabular[sort by name]{grade.csv}
\end{dispExample}
\end{docCommand}



\clearpage
\subsection{Data Collection}\label{sec:datacollection}

|csvsimple-l3| reads and processes a CSV file line by line. Accordingly, the \TeX{}
input stream is filled line by line.
Although this is an efficient procedure, for some applications like tables with
the \ctanpkg{tabularray} package, collecting the data from the CSV file into a macro is needed.
This macro can be given to the target application for further processing.


\begin{docCsvKey}[][doc new and updated={2021-07-06}{2023-10-17}]{collect data}{\colOpt{=true\textbar false}}{default |true|, initially |false|}
|csvsimple-l3| provides limited and experimental support to collect the input data
from the CSV file plus user additions into a macro named \refCom{csvdatacollection}.
Setting \refKey{csvsim/collect data} adds the contents of the following keys
to \refCom{csvdatacollection}:
\begin{itemize}
\item\refKey{csvsim/after head}
\item\refKey{csvsim/after first line}
\item\refKey{csvsim/after line}
\item\refKey{csvsim/before first line}
\item\refKey{csvsim/before line}
\item\refKey{csvsim/late after first line}
\item\refKey{csvsim/late after head}
\item\refKey{csvsim/late after last line}
\item\refKey{csvsim/late after line}
\end{itemize}
Also, the \emph{expanded} content of
\begin{itemize}
\item\refKey{csvsim/command}
\end{itemize}
is added to \refCom{csvdatacollection}
(depending on \refKey{csvsim/consume collected data} and \refKey{csvsim/data collection}).
Note that for \refKey{csvsim/command} special care has to be taken
\emph{what} should be protected from expansion and \emph{what not}.
Observe the following hints for \refKey{csvsim/command}:
\begin{itemize}
\item For data macros like |\csvcoli| use |\csvexpval\csvcoli| to add
  the \emph{value} of this macro to \refCom{csvdatacollection}.
  This is optional, if |\csvcoli| contains numbers or text without active
  characters, but essential, if it contains macros.
\item \refCom{csvlinetotablerow} is to be used \emph{without} |\csvexpval|.
\item For macros like |\textbf| use  |\csvexpnot\textbf| to \emph{prevent}
  expansion.
\item Using computations or not expandable conditionals may likely cause
  compilation errors.
\end{itemize}

\begin{dispExample}
\csvreader[
    collect data,
    head to column names,
    late after line=\\,
    late after last line=,
  ]{grade.csv}{}{%
    \thecsvrow. \csvexpval\givenname\ \csvexpnot\textbf{\csvexpval\name}
  }
Collected data:\par
\csvdatacollection
\end{dispExample}

Note that data collection is \emph{limited} to some special cases and does not
allow to save all possible content. Table options like \refKey{csvsim/longtable}
are generally not supported with the important exception of \refKey{csvsim/tabularray}
which uses \refKey{csvsim/collect data} automatically.\par
See \Fullref{sec:tabularray} for examples.
\end{docCsvKey}

\clearpage

\begin{docCsvKey}[][doc new={2023-12-18}]{consume collected data}{\colOpt{=true\textbar false}}{default |true|, initially |false|}
If set to |false|, the collected data of a CSV file processed with
\refKey{csvsim/collect data} is saved into \refCom{csvdatacollection}.\par
Otherwise, if set to |true|, the collected data is not saved, but directly used
after reading the CSV file, see \refKey{csvsim/generic collected table}.
After usage, the collected data is cleared, i.e. \refCom{csvdatacollection} is emptied.
\end{docCsvKey}


\begin{docCsvKey}[][doc new and updated={2021-07-06}{2024-05-16}]{data collection}{=\meta{macro}}{no default, initially \refCom{csvdatacollection}}
Sets the collection macro to an alternative for \refCom{csvdatacollection}.
\begin{dispListing}
  data collection = \myData,    %  instead of \csvdatacollection
\end{dispListing}
Note that until version 2.6.0 (2024/01/19), \refKey{csvsim/data collection} was
not reset to the default \refCom{csvdatacollection} for following CSV files,
but it is now.
\end{docCsvKey}


\begin{docCommand}[doc new=2021-07-06]{csvdatacollection}{}
  Macro which contains the collected data of a CSV file processed with
  \refKey{csvsim/collect data}. This macro name can be changed by
  setting \refKey{csvsim/data collection}.
\end{docCommand}


\begin{docCommand}[doc new and updated={2021-07-06}{2023-12-17}]{csvexpval}{\meta{macro}}
  Recovers the content of the given \meta{macro} and prevents further
  expansion. This is a wrapper for \docAuxCommand*{exp_not:o}.
  Alternatively, |\expandonce| from \ctanpkg{etoolbox} could be used.
\end{docCommand}


\begin{docCommand}[doc new=2021-07-06]{csvexpnot}{\meta{macro}}
  Prevents the expansion of the given \meta{macro}. This is a wrapper
  for \docAuxCommand*{exp_not:N}.
  Alternatively, |\noexpand| could be used.
\end{docCommand}

The following macros can only be used inside keys which are \emph{not}
collected to \refCom{csvdatacollection}, e.g. inside \refKey{csvsim/after filter}.

\begin{docCommand}[doc new and updated={2021-07-06}{2023-12-17}]{csvcollectn}{\marg{code}}
  Appends the given \meta{code} to \refCom{csvdatacollection}.\\
  This corresponds to \docAuxCommand*{tl_build_gput_right:Nn}.
\end{docCommand}


\begin{docCommands}[
      doc parameter = \marg{code}
    ]
  {
    { doc name = csvcollecte, doc new and updated = {2021-07-06}{2023-12-18} },
    { doc name = csvcollectx }
  }
  Appends the expansion of the given \meta{code} to \refCom{csvdatacollection}.\\
  This corresponds to \docAuxCommand*{tl_build_gput_right:Ne}.\\
  \refCom{csvcollectx} is an alias for \refCom{csvcollecte} and is kept for backward compatibility.
\end{docCommands}

\begin{docCommand}[doc new and updated={2021-07-06}{2023-12-17}]{csvcollectV}{\meta{macro}}
  Appends the content of the given \meta{macro} to \refCom{csvdatacollection}.\\
  This corresponds to \docAuxCommand*{tl_build_gput_right:Ne} and \docAuxCommand*{exp_not:o} for
  \meta{macro}.
\end{docCommand}


\clearpage
\section{String and Number Tests}\label{sec:stringtests}%

The following string and number tests are, to some extent, provided for
backward compatibility.
Mainly, they are wrappers for corresponding |expl3| conditionals.
Therefore, you are encouraged to use the following CamelCase macros
like \refCom{IfCsvsimStrEqualTF} which provide by their name insight
to the underlying |expl3| functions. The lowercase variants are kept
for backward compatibility.
\medskip


\begin{docCommands}[
      doc parameter = \marg{string A}\marg{string B}\marg{true}\marg{false}
    ]
  {
    { doc name = IfCsvsimStrEqualTF, doc new and updated = {2016-07-01}{2023-12-19} },
    { doc name = ifcsvstrcmp, color command=black }
  }
  Compares two strings and executes \meta{true} if they are equal, and \meta{false} otherwise.
  The comparison is done using |\str_if_eq:eeTF|.
  \refCom{IfCsvsimStrEqualTF} is expandable.
  Typically, this is the preferred function for many use cases.
\end{docCommands}


\begin{docCommand}[doc new and updated={2016-07-01}{2021-06-28},color command=black]{ifcsvnotstrcmp}{\marg{string A}\marg{string B}\marg{true}\marg{false}}
  Compares two strings and executes \meta{true} if they are \emph{not} equal, and \meta{false} otherwise.
  The implementation uses \refCom{IfCsvsimStrEqualTF}.
  \refCom{ifcsvnotstrcmp} is expandable.
  Consider using \refCom{IfCsvsimStrEqualTF} alternatively.
\end{docCommand}


\begin{docCommands}[
      doc parameter = \marg{token list A}\marg{token list B}\marg{true}\marg{false}
    ]
  {
    { doc name = IfCsvsimTlEqualTF, doc new and updated = {2016-07-01}{2023-12-19} },
    { doc name = ifcsvstrequal, color command=black }
  }
  Compares two token lists and executes \meta{true} if they are equal, and \meta{false} otherwise.
  The comparison is done using |\tl_if_eq:eeTF|.
  \refCom{IfCsvsimTlEqualTF} is not expandable.
  If you have no special reason for using a token list comparison, where
  characters and category codes of those characters are compared, you may
  rather choose \refCom{IfCsvsimStrEqualTF}.
\end{docCommands}



\begin{docCommands}[
      doc parameter = \marg{token list A}\marg{token list B}\marg{true}\marg{false}
    ]
  {
    { doc name = IfCsvsimTlProtectedEqualTF, doc new and updated = {2016-07-01}{2023-12-19} },
    { doc name = ifcsvprostrequal, color command=black }
  }
  Compares two token lists and executes \meta{true} if they are equal, and \meta{false} otherwise.
  The token lists are expanded with |\protected@edef|
  in the test, i.e. parts of the
  token lists which are protected stay unexpanded.
  The comparison is done using |\tl_if_eq:NNTF|.
  \refCom{IfCsvsimTlProtectedEqualTF} is not expandable.
\end{docCommands}


\begin{docCommands}[
      doc parameter = \marg{floating point comparison}\marg{true}\marg{false}
    ]
  {
    { doc name = IfCsvsimFpCompareTF, doc new and updated = {2021-06-28}{2023-12-19} },
    { doc name = ifcsvfpcmp, color command=black }
  }
  Evaluates the given \meta{floating point comparison}
  and executes \meta{true} or \meta{false} appropriately.
  The evaluation is done using \docAuxCommand*{fp_compare:nTF}.\\
  Basically, a \meta{floating point comparison} consists of
  \mbox{\meta{fp expr\textsubscript{1}} \meta{relation} \meta{fp expr\textsubscript{2}}},
  like \mbox{$x<y$}, but \docAuxCommand*{fp_compare:nTF} even allows a chain of comparisons.
  \refCom{IfCsvsimFpCompareTF} is expandable.
\end{docCommands}


\begin{docCommands}[
      doc parameter = \marg{integer comparison}\marg{true}\marg{false}
    ]
  {
    { doc name = IfCsvsimIntCompareTF, doc new and updated = {2021-06-28}{2023-12-19} },
    { doc name = ifcsvintcmp, color command=black }
  }
  Evaluates the given \meta{integer comparison}
  and executes \meta{true} or \meta{false} appropriately.
  The evaluation is done using \docAuxCommand*{int_compare:nTF}.\\
  Basically, a \meta{integer comparison} consists of
  \mbox{\meta{int expr\textsubscript{1}} \meta{relation} \meta{int expr\textsubscript{2}}},
  like \mbox{$x<y$}, but \docAuxCommand*{int_compare:nTF} even allows a chain of comparisons.
  \refCom{IfCsvsimIntCompareTF} is expandable.
\end{docCommands}



\clearpage
\section{Hooks}\label{sec:hooks}%
The following hook(s) are present following \LaTeX's hook management.

\begin{description}
\item[\textcolor{DarkViolet}{\ttfamily csvsimple/csvline}]
  \tcbdocmarginnote{\tcbdocnew{2023-05-08}}
  This hook adds code after reading
  a line into \refCom{csvline} and before processing this line.
  The token list \refCom{csvline} may be manipulated with a global assignment.\par
  The following example replaces every \verb+"..."+ by \verb+{...}+ to
  approximate double-quote processing within \LaTeX. Still, masking of double-quotes
  or nesting will not work.
\begin{dispListing}
\AddToHook{csvsimple/csvline}
  {
    \tl_set_eq:NN \l_tmpa_tl \csvline
    \regex_replace_all:nnN { "([^"]+)" } { {\1} } \l_tmpa_tl
    \tl_gset_eq:NN \csvline \l_tmpa_tl
  }
\end{dispListing}

\end{description}


\clearpage
\section{Examples}%

\subsection{A Serial Letter}%
In this example, a serial letter is to be written to all persons with
addresses from the following CSV file. Deliberately, the file content is
not given in very pretty format.

%-- file embedded for simplicity --
\begin{tcbverbatimwrite}{address.csv}
name,givenname,gender,degree,street,zip,location,bonus
Maier,Hans,m,,Am Bachweg 17,10010,Hopfingen,20
    % next line with a comma in curly braces
Huber,Erna,f,Dr.,{Moosstraße 32, Hinterschlag},10020,Örtingstetten,30
Weißbäck,Werner,m,Prof. Dr.,Brauallee 10,10030,Klingenbach,40
    % this line is ignored %
  Siebener ,  Franz,m,   ,  Blaumeisenweg 12  , 10040 ,  Pardauz , 50
    % preceding and trailing spaces in entries are removed %
Schmitt,Anton,m,,{\AE{}lfred-Esplanade, T\ae{}g 37}, 10050,\OE{}resung,60
\end{tcbverbatimwrite}
%-- end embedded file --

\csvlisting{address}

Firstly, we survey the file content quickly using
|\csvautotabular|.
As can be seen, unfeasible lines are ignored automatically.

\begin{dispExample}
\tiny\csvautotabular{address.csv}
\end{dispExample}

Now, we create the serial letter where every feasible data line produces
an own page. Here, we simulate the page by a |tcolorbox| (from the package
|tcolorbox|).
For the gender specific salutations, an auxiliary macro |\ifmale| is
introduced.

\begin{dispExample}
% this example requires the tcolorbox package
\newcommand{\ifmale}[2]{\IfCsvsimStrEqualTF{\gender}{m}{#1}{#2}}

\csvreader[head to column names]{address.csv}{}{%
\begin{tcolorbox}[colframe=DarkGray,colback=White,arc=0mm,width=(\linewidth-2pt)/2,
      equal height group=letter,before=,after=\hfill,fonttitle=\bfseries,
      adjusted title={Letter to \name}]
  \IfCsvsimStrEqualTF{\degree}{}{\ifmale{Mr.}{Ms.}}{\degree}~\givenname~\name\\
  \street\\\zip~\location
  \tcblower
  {\itshape Dear \ifmale{Sir}{Madam},}\\
  we are pleased to announce you a bonus value of \bonus\%{}
  which will be delivered to \location\ soon.\\\ldots
\end{tcolorbox}}
\end{dispExample}



\clearpage
\subsection{A Graphical Presentation}\label{sec:examgrapghpres}%
For this example, we use some artificial statistical data given by a CSV file.

%-- file embedded for simplicity --
\begin{tcbverbatimwrite}{data.csv}
land,group,amount
Bayern,A,1700
Baden-Württemberg,A,2300
Sachsen,B,1520
Thüringen,A,1900
Hessen,B,2100
\end{tcbverbatimwrite}
%-- end embedded file --

\csvlisting{data}

Firstly, we survey the file content using
|\csvautobooktabular|.

\begin{dispExample}
% needs the booktabs package
\csvautobooktabular{data.csv}
\end{dispExample}

The amount values are presented in the following diagram by bars where
the group classification is given using different colors.

\begin{dispExample}
% This example requires the package tikz
\begin{tikzpicture}[Group/A/.style={left color=red!10,right color=red!20},
                    Group/B/.style={left color=blue!10,right color=blue!20}]
\csvreader[head to column names]{data.csv}{}{%
  \begin{scope}[yshift=-\thecsvrow cm]
  \path [draw,Group/\group] (0,-0.45)
    rectangle node[font=\bfseries] {\amount} (\amount/1000,0.45);
  \node[left] at (0,0) {\land};
  \end{scope}  }
\end{tikzpicture}
\end{dispExample}


\clearpage
It would be nice to sort the bars by length, i.\,e.\ to sort the CSV file
by the |amount| column. If the \csvsorter\ program is properly installed,
see Subsection~\ref{sec:Sorting} on page~\pageref{sec:Sorting},
this can be done with the following configuration file for \csvsorter:

\xmllisting{amountsort}

Now, we just have to add an option |sort by=amountsort.xml|:
\begin{dispExample}
% This example requires the package tikz
% Also, the CSV-Sorter tool has to be installed
\begin{tikzpicture}[Group/A/.style={left color=red!10,right color=red!20},
                    Group/B/.style={left color=blue!10,right color=blue!20}]
\csvreader[head to column names,sort by=amountsort.xml]{data.csv}{}{%
  \begin{scope}[yshift=-\thecsvrow cm]
  \path [draw,Group/\group] (0,-0.45)
    rectangle node[font=\bfseries] {\amount} (\amount/1000,0.45);
  \node[left] at (0,0) {\land};
  \end{scope}  }
\end{tikzpicture}
\end{dispExample}




\clearpage
Next, we create a pie chart by calling |\csvreader| twice.
In the first step, the total sum of amounts is computed, and in the second
step the slices are drawn.

\begin{dispExample}
% Modified example from www.texample.net for pie charts
% This example needs the packages tikz, xcolor, calc
\definecolorseries{myseries}{rgb}{step}[rgb]{.95,.85,.55}{.17,.47,.37}
\resetcolorseries{myseries}%

% a pie slice
\newcommand{\slice}[4]{
  \pgfmathsetmacro{\midangle}{0.5*#1+0.5*#2}
  \begin{scope}
    \clip (0,0) -- (#1:1) arc (#1:#2:1) -- cycle;
    \colorlet{SliceColor}{myseries!!+}%
    \fill[inner color=SliceColor!30,outer color=SliceColor!60] (0,0) circle (1cm);
  \end{scope}
  \draw[thick] (0,0) -- (#1:1) arc (#1:#2:1) -- cycle;
  \node[label=\midangle:#4] at (\midangle:1) {};
  \pgfmathsetmacro{\temp}{min((#2-#1-10)/110*(-0.3),0)}
  \pgfmathsetmacro{\innerpos}{max(\temp,-0.5) + 0.8}
  \node at (\midangle:\innerpos) {#3};
}

% sum of amounts
\csvreader[before reading=\def\mysum{0}]{data.csv}{amount=\amount}{%
  \pgfmathsetmacro{\mysum}{\mysum+\amount}%
}

% drawing of the pie chart
\begin{tikzpicture}[scale=3]%
\def\mya{0}\def\myb{0}
\csvreader[head to column names]{data.csv}{}{%
  \let\mya\myb
  \pgfmathsetmacro{\myb}{\myb+\amount}
  \slice{\mya/\mysum*360}{\myb/\mysum*360}{\amount}{\land}
}
\end{tikzpicture}%
\end{dispExample}


\clearpage
Finally, the filter option is demonstrated by separating the groups A and B.
Every item is piled upon the appropriate stack.

\begin{dispExample}
\newcommand{\drawGroup}[2]{%
  \def\mya{0}\def\myb{0}
  \node[below=3mm] at (2.5,0) {\bfseries Group #1};
  \csvreader[head to column names,filter equal={\group}{#1}]{data.csv}{}{%
    \let\mya\myb
    \pgfmathsetmacro{\myb}{\myb+\amount}
    \path[draw,top color=#2!25,bottom color=#2!50]
      (0,\mya/1000) rectangle node{\land\ (\amount)} (5,\myb/1000);
}}

\begin{tikzpicture}
  \fill[gray!75] (-1,0) rectangle (13,-0.1);
  \drawGroup{A}{red}
  \begin{scope}[xshift=7cm]
  \drawGroup{B}{blue}
  \end{scope}
\end{tikzpicture}

\end{dispExample}


\clearpage
\subsection{Macro code inside the data}\label{macrocodexample}%

If needed, the data file may contain macro code.

%-- file embedded for simplicity --
\begin{tcbverbatimwrite}{macrodata.csv}
type,description,content
M,A nice \textbf{formula},         $\displaystyle \int\frac{1}{x} = \ln|x|+c$
G,A \textcolor{red}{colored} ball, {\tikz \shadedraw [shading=ball] (0,0) circle (.5cm);}
M,\textbf{Another} formula,        $\displaystyle \lim\limits_{n\to\infty} \frac{1}{n}=0$
\end{tcbverbatimwrite}
%-- end embedded file --

\csvlisting{macrodata}

Firstly, we survey the file content using
|\csvautobooktabular|.

\begin{dispExample}
\csvautobooktabular{macrodata.csv}
\end{dispExample}


\begin{dispExample}
\csvstyle{my enumerate}{head to column names,
  before reading=\begin{enumerate},after reading=\end{enumerate}}

\csvreader[my enumerate]{macrodata.csv}{}{%
  \item \description:\par\content}

\bigskip
Now, formulas only:
\csvreader[my enumerate,filter strcmp={\type}{M}]{macrodata.csv}{}{%
  \item \description:\qquad\content}
\end{dispExample}

\clearpage
\subsection{Tables with Number Formatting}\label{numberformatting}%

We consider a file with numerical data which should be pretty-printed.

%-- file embedded for simplicity --
\begin{tcbverbatimwrite}{data_numbers.csv}
month,    dogs, cats
January,  12.50,12.3e5
February, 3.32, 8.7e3
March,    43,   3.1e6
April,    0.33, 21.2e4
May,      5.12, 3.45e6
June,     6.44, 6.66e6
July,     123.2,7.3e7
August,   12.3, 5.3e4
September,2.3,  4.4e4
October,  6.5,  6.5e6
November, 0.55, 5.5e5
December, 2.2,  3.3e3
\end{tcbverbatimwrite}

\csvlisting{data_numbers}

\medskip

The \ctanpkg{siunitx} package provides a huge amount of formatting options for
numbers. A good and robust way to apply formatting by \ctanpkg{siunitx} inside
tables generated by |csvsimple-l3| is the |\tablenum| macro from
\ctanpkg{siunitx}.

\begin{dispExample}
% \usepackage{siunitx,array,booktabs}
\csvreader[
    head to column names,
    before reading = \begin{center}\sisetup{table-number-alignment=center},
    tabular        = cc,
    table head     = \toprule \textbf{Cats} & \textbf{Dogs} \\\midrule,
    table foot     = \bottomrule,
    after reading  = \end{center}
  ]{data_numbers.csv}{}{%
    \tablenum[table-format=2.2e1]{\cats} & \tablenum{\dogs}
  }
\end{dispExample}

\clearpage

It is also possible to create on-the-fly tables using calculations of
the given data. The following example shows cat values bisected and
dog values doubled.

\begin{dispExample}
% \usepackage{siunitx,array,booktabs,xfp}
\csvreader[
    head to column names,
    before reading = \begin{center}\sisetup{table-number-alignment=center},
    tabular        = cccc,
    table head     = \toprule \textbf{Cats} & \textbf{Dogs}
                     & \textbf{Halfcats} & \textbf{Doubledogs} \\\midrule,
    table foot     = \bottomrule,
    after reading  = \end{center}
  ]{data_numbers.csv}{}{%
    \tablenum[table-format=2.2e1]{\cats} & \tablenum{\dogs}
      & \tablenum[exponent-mode=scientific, round-precision=3,
          round-mode=places, table-format=1.3e1]{\fpeval{\cats/2}}
      & \tablenum{\fpeval{\dogs*2}}
  }
\end{dispExample}


\clearpage

The |siunitx| package also provides a new column type |S|
which can align material using a number of different strategies.
Special care is needed, if the \emph{first} or the \emph{last} column is to be formatted with
the column type |S|. The number detection of |siunitx| is disturbed by
the line reading code of |csvsimple-l3| which actually is present at the
first and last column. To avoid this problem, the utilization of
|\tablenum| is appropriate, see above.
Alternatively, a very nifty workaround suggested by Enrico Gregorio is to
add an invisible dummy column with |c@{}| as first column
and |@{}c| as last column:

\begin{dispExample}
% \usepackage{siunitx,array,booktabs}
\csvreader[
    head to column names,
    before reading = \begin{center}\sisetup{table-number-alignment=center},
    tabular        = {c@{}S[table-format=2.2e1]S@{}c},
    table head     = \toprule & \textbf{Cats} & \textbf{Dogs} & \\\midrule,
    table foot     = \bottomrule,
    after reading  = \end{center}
  ]{data_numbers.csv}{}{%
    & \cats & \dogs &
  }
\end{dispExample}




\clearpage
Now, the preceding table shall be sorted by the \emph{cats} values.
If the \csvsorter\ program is properly installed,
see Subsection~\ref{sec:Sorting} on page~\pageref{sec:Sorting},
this can be done with the following configuration file for \csvsorter:

\xmllisting{catsort}

Now, we just have to add an option |sort by=catsort.xml|:
\begin{dispExample}
% \usepackage{siunitx,array,booktabs}
% Also, the CSV-Sorter tool has to be installed
\csvreader[
    head to column names,
    sort by        = catsort.xml,
    before reading = \begin{center}\sisetup{table-number-alignment=center},
    tabular        = lcc,
    table head     = \toprule \textbf{Month} & \textbf{Dogs} & \textbf{Cats} \\\midrule,
    table foot     = \bottomrule,
    after reading  = \end{center}
  ]{data_numbers.csv}{}{%
    \month & \tablenum{\dogs} & \tablenum[table-format=2.2e1]{\cats}
  }
\end{dispExample}


\clearpage
\subsection{CSV data without header line}\label{noheader}%
CSV files with a header line are more semantic than files without header,
but it's no problem to work with headless files.

For this example, we use again some artificial statistical data given by a CSV file
but this time without header.

%-- file embedded for simplicity --
\begin{tcbverbatimwrite}{data_headless.csv}
Bayern,A,1700
Baden-Württemberg,A,2300
Sachsen,B,1520
Thüringen,A,1900
Hessen,B,2100
\end{tcbverbatimwrite}
%-- end embedded file --

\csvlisting{data_headless}

Note that you cannot use the \refKey{csvsim/no head} option for the auto tabular
commands.
If no options are given, the first line is interpreted as header line
which gives an unpleasant result:

\begin{dispExample}
\csvautobooktabular{data_headless.csv}
\end{dispExample}

To get the expected result, the \emph{star} versions of the auto tabular
commands can be used.

\begin{dispExample}
\csvautobooktabular*{data_headless.csv}
\end{dispExample}

This example can be extended to insert a table head for this headless data:

\begin{dispExample}
\csvautobooktabular*[
    table head=\toprule\bfseries Land & \bfseries Group
                  & \bfseries Amount\\\midrule
  ]{data_headless.csv}
\end{dispExample}


\clearpage

For the normal \refCom{csvreader} command, the \refKey{csvsim/no head} option
should be applied. Of course, we cannot use \refKey{csvsim/head to column names}
because there is no head, but the columns can be addressed by their numbers:

\begin{dispExample}
\csvreader[
    no head,
    tabular    = lr,
    table head = \toprule\bfseries Land & \bfseries Amount\\\midrule,
    table foot = \bottomrule]
  {data_headless.csv}
  { 1=\land, 3=\amount }
  {\land & \amount}
\end{dispExample}



\clearpage
\subsection{Tables with \texttt{tabularray}}\label{sec:tabularray}%

The \ctanpkg{tabularray} package gives extended control for generating
tables. \refKey{csvsim/tabularray} and \refKey{csvsim/centered tabularray}
support such tables. A distinctiveness is that for \ctanpkg{tabularray}
data from a CSV file has to be \emph{collected} first (into a macro)
and applied afterwards. The process is hidden from the user view, but
has to be taken into account when \refKey{csvsim/command} is set up,
see \Fullref{sec:datacollection}.

The following examples uses |data.csv| from \Fullref{sec:examgrapghpres}.

\begin{dispExample}
% \usepackage{tabularray,siunitx,xfp}
\csvreader[
    head to column names,
    centered tabularray =
      {
        rowsep = 1mm,
        colsep = 5mm,
        rows   = {blue7},
        hlines = {2pt, white},
        vlines = {2pt, white},
        row{1} = {bg=azure3, fg=white, font=\bfseries\large, 8mm},
      },
    table head = {\SetCell[c=4]{c} Important Data Table \\},
  ]{data.csv}{}{
       \IfCsvsimStrEqualTF{\group}{A}{\csvexpnot\SetRow{brown7}}{}
       \csvexpnot\SetCell{bg=purple7}
       \csvexpval\land
     & \csvexpval\group
     & \csvexpval\amount
     & \tablenum[exponent-mode=scientific, round-precision=3,
         round-mode=places, table-format=1.3e1]{\fpeval{pi*\amount}}
  }
\end{dispExample}

Note in the example above that
\begin{itemize}
\item \refKey{csvsim/table head} is \emph{collected} unexpanded, i.e.
  |\SetCell| has not to be protected. On the other hand, CSV data could not
  be used here.
\item \refKey{csvsim/command} is \emph{collected} expanded. This is identical
  to the mandatory last argument of \refCom{csvreader}.
  \begin{itemize}
  \item Therefore, expansion of |\SetRow|, |\SetCell|, etc. is prevented by \refCom{csvexpnot}.
  \item The \emph{values} (content) of |\land|, |\group|, etc. are recovered by
        \refCom{csvexpval}.
  \item |\IfCsvsimStrEqualTF| and |\fpeval| are \emph{expandable} and therefore the
    results of these commands are \emph{collected}.
  \item |\tablenum| from \ctanpkg{siunitx} is a robust command and therefore
    needs no protection. If you are not sure, if a command is robust or not, it
    does not hurt add the prefix \refCom{csvexpnot}, i.e. use |\csvexpnot\tablenum|.
  \end{itemize}
\end{itemize}

\clearpage

Filters and line ranges can be used for \ctanpkg{tabularray} and all
data collections without restriction:

\begin{dispExample}
% \usepackage{tabularray}
Display group `A` only:\par
\csvreader[
    head to column names,
    filter strcmp = {\group}{A},
    centered tabularray =
      {
        rowsep = 1mm,
        colsep = 5mm,
        column{1} = {r, fg=yellow5, colsep=2pt},
        column{2} = {r, yellow8!10, font=\bfseries},
        column{3} = {l, yellow8},
        hlines    = {2pt, white},
      },
  ]{data.csv}{}{
       \thecsvrow
     & \csvexpval\land
     & \csvexpval\amount
  }
\end{dispExample}


\begin{dispExample}
% \usepackage{tabularray}
Display data from line 3 on:\par
\csvreader[
    head to column names,
    range = 3-,
    centered tabularray =
      {
        rowsep = 1mm,
        colsep = 5mm,
        column{1} = {r, fg=violet5, colsep=2pt},
        column{2} = {r, violet8!10, font=\bfseries},
        column{3} = {l, violet8},
        hlines    = {2pt, white},
      },
  ]{data.csv}{}{
       \thecsvrow
     & \csvexpval\land
     & \csvexpval\amount
  }
\end{dispExample}


\clearpage

The following example uses \refCom{csvautotabularray} to display the whole
table. Note that the \ctanpkg{tabularray} options are given as last
optional argument.

\begin{dispExample}
% \usepackage{tabularray}
\csvautotabularray[table centered]{data.csv}
  [
    row{odd}   = {blue!85!gray!7},
    row{1}     = {blue!50!gray!25, font=\bfseries, preto=\MakeUppercase},
    hline{1,Z} = {0.1em, blue!50!black},
    hline{2}   = {blue!50!black}
  ]
\end{dispExample}



\clearpage
\subsection{Imported CSV data}\label{sec:importeddata}%
If data is imported from other applications, there is not always a choice
to format in comma separated values with curly brackets.

Consider the following example data file:

%-- file embedded for simplicity --
\begin{tcbverbatimwrite}{imported.csv}
"name";"address";"email"
"Frank Smith";"Yellow Road 123, Brimblsby";"frank.smith@organization.org"
"Mary May";"Blue Alley 2a, London";"mmay@maybe.uk"
"Hans Meier";"Hauptstraße 32, Berlin";"hans.meier@corporation.de"
\end{tcbverbatimwrite}
%-- end embedded file --

\csvlisting{imported}

If the \csvsorter\ program is properly installed,
see Subsection~\ref{sec:Sorting} on page~\pageref{sec:Sorting},
this can be transformed on-the-fly
with the following configuration file for \csvsorter:

\xmllisting{transform}

Now, we just have to add an option |sort by=transform.xml| to transform
the input data. Here, we actually do not sort.

\begin{dispExample}
% \usepackage{booktabs,array}
% Also, the CSV-Sorter tool has to be installed
\newcommand{\Header}[1]{\normalfont\bfseries #1}

\csvreader[
    sort by    = transform.xml,
    tabular    = >{\itshape}ll>{\ttfamily}l,
    table head = \toprule\Header{Name} & \Header{Address} & \Header{email}\\\midrule,
    table foot = \bottomrule
  ]
  {imported.csv}{}
  {\csvlinetotablerow}
\end{dispExample}

The file which is generated on-the-fly and which is actually read by
|csvsimple-l3| is the following:

\tcbinputlisting{docexample,listing style=tcbdocumentation,fonttitle=\bfseries,
  listing only,listing file=\jobname_sorted._csv}


\clearpage
\subsection{Encoding}\label{encoding}%
If the CSV file has a different encoding than the \LaTeX\ source file,
then special care is needed.

\begin{itemize}
\item The most obvious treatment is to change the encoding of the CSV file
  or the \LaTeX\ source file to match the other one (every good editor
  supports such a conversion). This is the easiest choice, if there a no
  good reasons against such a step. E.g., unfortunately, several tools
  under Windows need the CSV file to be |cp1252| encoded while
  the \LaTeX\ source file may need to be |utf8| encoded.

\item The |inputenc| package allows to switch the encoding inside the
  document, say from |utf8| to |cp1252|. Just be aware that you should only
  use pure ASCII for additional texts inside the switched region.
\begin{dispListing}
% !TeX encoding=UTF-8
% ....
\usepackage[utf8]{inputenc}
% ....
\begin{document}
% ....
\inputencoding{latin1}% only use ASCII from here, e.g. "Uberschrift
\csvreader[%...
  ]{data_cp1252.csv}{%...
  }{% ....
  }
\inputencoding{utf8}
% ....
\end{document}
\end{dispListing}

\item As a variant to the last method, the encoding switch can be done
  using options from |csvsimple-l3|:
\begin{dispListing}
% !TeX encoding=UTF-8
% ....
\usepackage[utf8]{inputenc}
% ....
\begin{document}
% ....
% only use ASCII from here, e.g. "Uberschrift
\csvreader[%...
  before reading=\inputencoding{latin1},
  after reading=\inputencoding{utf8},
  ]{data_cp1252.csv}{%...
  }{% ....
  }
% ....
\end{document}
\end{dispListing}

\pagebreak\item
If the \csvsorter\ program is properly installed,
see Subsection~\ref{sec:Sorting} on page~\pageref{sec:Sorting},
the CSV file can be re-encoded on-the-fly
with the following configuration file for \csvsorter:

\xmllisting{encoding}

\begin{dispListing}
% !TeX encoding=UTF-8
% ....
\usepackage[utf8]{inputenc}
% ....
\begin{document}
% ....
\csvreader[%...
  sort by=encoding.xml,
  ]{data_cp1252.csv}{%...
  }{% ....
  }
% ....
\end{document}
\end{dispListing}


\end{itemize}


\clearpage

\printindex

\end{document}
